\documentclass[11pt,twoside]{article}\makeatletter

\IfFileExists{xcolor.sty}%
  {\RequirePackage{xcolor}}%
  {\RequirePackage{color}}
\usepackage{colortbl}
\usepackage{wrapfig}
\usepackage{ifxetex}
\ifxetex
  \usepackage{fontspec}
  \usepackage{xunicode}
  \catcode`⃥=\active \def⃥{\textbackslash}
  \catcode`❴=\active \def❴{\{}
  \catcode`❵=\active \def❵{\}}
  \def\textJapanese{\fontspec{IPAMincho}}
  \def\textChinese{\fontspec{HAN NOM A}\XeTeXlinebreaklocale "zh"\XeTeXlinebreakskip = 0pt plus 1pt }
  \def\textKorean{\fontspec{Baekmuk Gulim} }
  \setmonofont{DejaVu Sans Mono}
  
\else
  \IfFileExists{utf8x.def}%
   {\usepackage[utf8x]{inputenc}
      \PrerenderUnicode{–}
    }%
   {\usepackage[utf8]{inputenc}}
  \usepackage[english]{babel}
  \usepackage[T1]{fontenc}
  \usepackage{float}
  \usepackage[]{ucs}
  \uc@dclc{8421}{default}{\textbackslash }
  \uc@dclc{10100}{default}{\{}
  \uc@dclc{10101}{default}{\}}
  \uc@dclc{8491}{default}{\AA{}}
  \uc@dclc{8239}{default}{\,}
  \uc@dclc{20154}{default}{ }
  \uc@dclc{10148}{default}{>}
  \def\textschwa{\rotatebox{-90}{e}}
  \def\textJapanese{}
  \def\textChinese{}
  \IfFileExists{tipa.sty}{\usepackage{tipa}}{}
  \usepackage{times}
\fi
\def\exampleFont{\ttfamily\small}
\DeclareTextSymbol{\textpi}{OML}{25}
\usepackage{relsize}
\RequirePackage{array}
\def\@testpach{\@chclass
 \ifnum \@lastchclass=6 \@ne \@chnum \@ne \else
  \ifnum \@lastchclass=7 5 \else
   \ifnum \@lastchclass=8 \tw@ \else
    \ifnum \@lastchclass=9 \thr@@
   \else \z@
   \ifnum \@lastchclass = 10 \else
   \edef\@nextchar{\expandafter\string\@nextchar}%
   \@chnum
   \if \@nextchar c\z@ \else
    \if \@nextchar l\@ne \else
     \if \@nextchar r\tw@ \else
   \z@ \@chclass
   \if\@nextchar |\@ne \else
    \if \@nextchar !6 \else
     \if \@nextchar @7 \else
      \if \@nextchar (8 \else
       \if \@nextchar )9 \else
  10
  \@chnum
  \if \@nextchar m\thr@@\else
   \if \@nextchar p4 \else
    \if \@nextchar b5 \else
   \z@ \@chclass \z@ \@preamerr \z@ \fi \fi \fi \fi
   \fi \fi  \fi  \fi  \fi  \fi  \fi \fi \fi \fi \fi \fi}
\gdef\arraybackslash{\let\\=\@arraycr}
\def\@textsubscript#1{{\m@th\ensuremath{_{\mbox{\fontsize\sf@size\z@#1}}}}}
\def\Panel#1#2#3#4{\multicolumn{#3}{){\columncolor{#2}}#4}{#1}}
\def\abbr{}
\def\corr{}
\def\expan{}
\def\gap{}
\def\orig{}
\def\reg{}
\def\ref{}
\def\sic{}
\def\persName{}\def\name{}
\def\placeName{}
\def\orgName{}
\def\textcal#1{{\fontspec{Lucida Calligraphy}#1}}
\def\textgothic#1{{\fontspec{Lucida Blackletter}#1}}
\def\textlarge#1{{\large #1}}
\def\textoverbar#1{\ensuremath{\overline{#1}}}
\def\textquoted#1{‘#1’}
\def\textsmall#1{{\small #1}}
\def\textsubscript#1{\@textsubscript{\selectfont#1}}
\def\textxi{\ensuremath{\xi}}
\def\titlem{\itshape}
\newenvironment{biblfree}{}{\ifvmode\par\fi }
\newenvironment{bibl}{}{}
\newenvironment{byline}{\vskip6pt\itshape\fontsize{16pt}{18pt}\selectfont}{\par }
\newenvironment{citbibl}{}{\ifvmode\par\fi }
\newenvironment{docAuthor}{\ifvmode\vskip4pt\fontsize{16pt}{18pt}\selectfont\fi\itshape}{\ifvmode\par\fi }
\newenvironment{docDate}{}{\ifvmode\par\fi }
\newenvironment{docImprint}{\vskip 6pt}{\ifvmode\par\fi }
\newenvironment{docTitle}{\vskip6pt\bfseries\fontsize{18pt}{22pt}\selectfont}{\par }
\newenvironment{msHead}{\vskip 6pt}{\par}
\newenvironment{msItem}{\vskip 6pt}{\par}
\newenvironment{rubric}{}{}
\newenvironment{titlePart}{}{\par }

\newcolumntype{L}[1]{){\raggedright\arraybackslash}p{#1}}
\newcolumntype{C}[1]{){\centering\arraybackslash}p{#1}}
\newcolumntype{R}[1]{){\raggedleft\arraybackslash}p{#1}}
\newcolumntype{P}[1]{){\arraybackslash}p{#1}}
\newcolumntype{B}[1]{){\arraybackslash}b{#1}}
\newcolumntype{M}[1]{){\arraybackslash}m{#1}}
\definecolor{label}{gray}{0.75}
\def\unusedattribute#1{\sout{\textcolor{label}{#1}}}
\DeclareRobustCommand*{\xref}{\hyper@normalise\xref@}
\def\xref@#1#2{\hyper@linkurl{#2}{#1}}
\begingroup
\catcode`\_=\active
\gdef_#1{\ensuremath{\sb{\mathrm{#1}}}}
\endgroup
\mathcode`\_=\string"8000
\catcode`\_=12\relax

\usepackage[a4paper,twoside,lmargin=1in,rmargin=1in,tmargin=1in,bmargin=1in,marginparwidth=0.75in]{geometry}
\usepackage{framed}

\definecolor{shadecolor}{gray}{0.95}
\usepackage{longtable}
\usepackage[normalem]{ulem}
\usepackage{fancyvrb}
\usepackage{fancyhdr}
\usepackage{graphicx}
\usepackage{marginnote}


\renewcommand*{\marginfont}{\itshape\footnotesize}

\def\Gin@extensions{.pdf,.png,.jpg,.mps,.tif}

  \pagestyle{fancy}

\usepackage[pdftitle={Vida de Aníbal, escrita por Donato Acciaiuoli y traducida por Alfonso de Palencia},
 pdfauthor={}]{hyperref}
\hyperbaseurl{}

\usepackage[noreledmac]{eledmac}

\renewcommand{\notenumfont}{\bfseries}
\lineation{page}
\linenummargin{inner}
\footthreecol{A}
\foottwocol{B}

	 \paperwidth210mm
	 \paperheight297mm
              
\def\@pnumwidth{1.55em}
\def\@tocrmarg {2.55em}
\def\@dotsep{4.5}
\setcounter{tocdepth}{3}
\clubpenalty=8000
\emergencystretch 3em
\hbadness=4000
\hyphenpenalty=400
\pretolerance=750
\tolerance=2000
\vbadness=4000
\widowpenalty=10000

\renewcommand\section{\@startsection {section}{1}{\z@}%
     {-1.75ex \@plus -0.5ex \@minus -.2ex}%
     {0.5ex \@plus .2ex}%
     {\reset@font\Large\bfseries\sffamily}}
\renewcommand\subsection{\@startsection{subsection}{2}{\z@}%
     {-1.75ex\@plus -0.5ex \@minus- .2ex}%
     {0.5ex \@plus .2ex}%
     {\reset@font\Large\sffamily}}
\renewcommand\subsubsection{\@startsection{subsubsection}{3}{\z@}%
     {-1.5ex\@plus -0.35ex \@minus -.2ex}%
     {0.5ex \@plus .2ex}%
     {\reset@font\large\sffamily}}
\renewcommand\paragraph{\@startsection{paragraph}{4}{\z@}%
     {-1ex \@plus-0.35ex \@minus -0.2ex}%
     {0.5ex \@plus .2ex}%
     {\reset@font\normalsize\sffamily}}
\renewcommand\subparagraph{\@startsection{subparagraph}{5}{\parindent}%
     {1.5ex \@plus1ex \@minus .2ex}%
     {-1em}%
     {\reset@font\normalsize\bfseries}}


\def\l@section#1#2{\addpenalty{\@secpenalty} \addvspace{1.0em plus 1pt}
 \@tempdima 1.5em \begingroup
 \parindent \z@ \rightskip \@pnumwidth 
 \parfillskip -\@pnumwidth 
 \bfseries \leavevmode #1\hfil \hbox to\@pnumwidth{\hss #2}\par
 \endgroup}
\def\l@subsection{\@dottedtocline{2}{1.5em}{2.3em}}
\def\l@subsubsection{\@dottedtocline{3}{3.8em}{3.2em}}
\def\l@paragraph{\@dottedtocline{4}{7.0em}{4.1em}}
\def\l@subparagraph{\@dottedtocline{5}{10em}{5em}}
\@ifundefined{c@section}{\newcounter{section}}{}
\@ifundefined{c@chapter}{\newcounter{chapter}}{}
\newif\if@mainmatter 
\@mainmattertrue
\def\chaptername{Chapter}
\def\frontmatter{%
  \pagenumbering{roman}
  \def\thechapter{\@roman\c@chapter}
  \def\theHchapter{\roman{chapter}}
  \def\thesection{\@roman\c@section}
  \def\theHsection{\roman{section}}
  \def\@chapapp{}%
}
\def\mainmatter{%
  \cleardoublepage
  \def\thechapter{\@arabic\c@chapter}
  \setcounter{chapter}{0}
  \setcounter{section}{0}
  \pagenumbering{arabic}
  \setcounter{secnumdepth}{6}
  \def\@chapapp{\chaptername}%
  \def\theHchapter{\arabic{chapter}}
  \def\thesection{\@arabic\c@section}
  \def\theHsection{\arabic{section}}
}
\def\backmatter{%
  \cleardoublepage
  \setcounter{chapter}{0}
  \setcounter{section}{0}
  \setcounter{secnumdepth}{2}
  \def\@chapapp{\appendixname}%
  \def\thechapter{\@Alph\c@chapter}
  \def\theHchapter{\Alph{chapter}}
  \appendix
}
\newenvironment{bibitemlist}[1]{%
   \list{\@biblabel{\@arabic\c@enumiv}}%
       {\settowidth\labelwidth{\@biblabel{#1}}%
        \leftmargin\labelwidth
        \advance\leftmargin\labelsep
        \@openbib@code
        \usecounter{enumiv}%
        \let\p@enumiv\@empty
        \renewcommand\theenumiv{\@arabic\c@enumiv}%
	}%
  \sloppy
  \clubpenalty4000
  \@clubpenalty \clubpenalty
  \widowpenalty4000%
  \sfcode`\.\@m}%
  {\def\@noitemerr
    {\@latex@warning{Empty `bibitemlist' environment}}%
    \endlist}

\def\tableofcontents{\section*{\contentsname}\@starttoc{toc}}
\parskip0pt
\parindent1em
\def\Panel#1#2#3#4{\multicolumn{#3}{){\columncolor{#2}}#4}{#1}}
\newenvironment{reflist}{%
  \begin{raggedright}\begin{list}{}
  {%
   \setlength{\topsep}{0pt}%
   \setlength{\rightmargin}{0.25in}%
   \setlength{\itemsep}{0pt}%
   \setlength{\itemindent}{0pt}%
   \setlength{\parskip}{0pt}%
   \setlength{\parsep}{2pt}%
   \def\makelabel##1{\itshape ##1}}%
  }
  {\end{list}\end{raggedright}}
\newenvironment{sansreflist}{%
  \begin{raggedright}\begin{list}{}
  {%
   \setlength{\topsep}{0pt}%
   \setlength{\rightmargin}{0.25in}%
   \setlength{\itemindent}{0pt}%
   \setlength{\parskip}{0pt}%
   \setlength{\itemsep}{0pt}%
   \setlength{\parsep}{2pt}%
   \def\makelabel##1{\upshape\sffamily ##1}}%
  }
  {\end{list}\end{raggedright}}
\newenvironment{specHead}[2]%
 {\vspace{20pt}\hrule\vspace{10pt}%
  \label{#1}\markright{#2}%

  \pdfbookmark[2]{#2}{#1}%
  \hspace{-0.75in}{\bfseries\fontsize{16pt}{18pt}\selectfont#2}%
  }{}
      \def\TheFullDate{2017-07-26 (revised: Junio, 2016)}
\def\TheID{\makeatother }
\def\TheDate{2017-07-26}
\title{Vida de Aníbal, escrita por Donato Acciaiuoli y traducida por Alfonso de Palencia}
\author{}\makeatletter 
\makeatletter
\newcommand*{\cleartoleftpage}{%
  \clearpage
    \if@twoside
    \ifodd\c@page
      \hbox{}\newpage
      \if@twocolumn
        \hbox{}\newpage
      \fi
    \fi
  \fi
}
\makeatother
\makeatletter
\thispagestyle{empty}
\markright{\@title}\markboth{\@title}{\@author}
\renewcommand\small{\@setfontsize\small{9pt}{11pt}\abovedisplayskip 8.5\p@ plus3\p@ minus4\p@
\belowdisplayskip \abovedisplayskip
\abovedisplayshortskip \z@ plus2\p@
\belowdisplayshortskip 4\p@ plus2\p@ minus2\p@
\def\@listi{\leftmargin\leftmargini
               \topsep 2\p@ plus1\p@ minus1\p@
               \parsep 2\p@ plus\p@ minus\p@
               \itemsep 1pt}
}
\makeatother
\fvset{frame=single,numberblanklines=false,xleftmargin=5mm,xrightmargin=5mm}
\fancyhf{} 
\setlength{\headheight}{14pt}
\fancyhead[LE]{\bfseries\leftmark} 
\fancyhead[RO]{\bfseries\rightmark} 
\fancyfoot[RO]{}
\fancyfoot[CO]{\thepage}
\fancyfoot[LO]{\TheID}
\fancyfoot[LE]{}
\fancyfoot[CE]{\thepage}
\fancyfoot[RE]{\TheID}
\hypersetup{linkbordercolor=0.75 0.75 0.75,urlbordercolor=0.75 0.75 0.75,bookmarksnumbered=true}
\fancypagestyle{plain}{\fancyhead{}\renewcommand{\headrulewidth}{0pt}}\makeatother 
\begin{document}

\makeatletter
\noindent\parbox[b]{.75\textwidth}{\fontsize{14pt}{16pt}\bfseries\raggedright\sffamily\selectfont \@title}
\vskip20pt
\par\noindent{\fontsize{11pt}{13pt}\sffamily\itshape\raggedright\selectfont\@author\hfill\TheDate}
\vspace{18pt}
\makeatother
\let\tabcellsep& 
\beginnumbering
\def\endstanzaextra{\pstart\centering---------\skipnumbering\pend}
\pstart
 {\persName Plutarcho}  {\reg philósopho}  {\reg escrivió} en griego la vida del ylustre  {\reg varón}  {\persName Hanníbal};  {\reg bolviola} en  {\reg latín}  {\persName Donato de Acciayolo} florentino y el cronista  {\persName Alfonso de Palencia} la traduxo en romançe castellano.
\pend
\pstart
1.1% image:/Vitae/public/images/1491/165v.png
[165v,a] Si se repite la memoria de la Primera Guerra Púnica que los  {\name carthagineses} fizieron con el  {\name pueblo romano}, nos occurrirán muchos capitanes que, aviendo conseguido gloria de sus fazañas, dexaron a sus successores honrosa fama. 2 Mas ninguno hay de los capitanes  {\name carthagineses} que los griegos y latinos scriptores más ensalçen que a  {\persName Hamílcar}, padre de  {\persName Hanníbal}, renombrado  {\name Barcha}, varón, sin dubda, principal y muy enseñado en el exercito militar como podiera aver algún otro en aquella edad. 3 Aqueste, más luengamente \edtext{que}{\Afootnote{; qua ()}} la razón quería, sostovo primero en  {\placeName Sicilia} el ímpeto de los  {\name romanos}, que avían induzido muchas pérdidas y rompimientos a su república carthaginesa. 4 Después, en la guerra africana, quando los guerreros soldados se levantaron con escándalo y posieron la \edtext{cosa}{\Afootnote{; casa ()}} pública de los  {\name carthagineses} en extremo peligro, él apagó tan strenuamente aquel fuego, que por sentençia de todos es juzgado aver sido conservada la patria en aquel tiempo por obra de un solo  {\persName Hamílcar}. 5 Después d'esto, en  {\placeName España} embiado con capitanía quando se cuenta aver levado consigo a  {\persName Hanníbal} mochacho, fechas muchas cosas dignas de memoria, al cabo en el noveno año después de venido en aquella provincia, murió peleando % image:../public/images/1491/165v.png
[165v,b] fuertemente contra los  {\name vetheones}.
\pend
\pstart
2.1 Muerto  {\persName Hamílcar}, su yerno  {\persName Hasdrúbal}, que los  {\name carthagineses} con favor del vando barchino prefirieron al exército, tovo la capitanía quasi ocho años. El qual fizo venir en  {\placeName España} a  {\persName Hanníbal} no mucho tiempo después de la muerte del padre, aunque lo contradizían los principales del otro vando, y llamolo porque, segund avía començado {\small\itshape [1P. omite el lat. \textit{puer}]}  en la vida del padre {\small\itshape [2padre: en lugar del lat. \textit{Hamilchare}]}  a ser enseñado en las artes de la guerra, assí aún entonçes, en edad más robusta, se diesse a los peligros y trabajos y a todos officios militares. 2 Y comoquier que desd'el comienço la memoria del padre no poco le pudo aprovechar para ganar favor en el exército, pero dende a poco su obra e ingenio fue tal, que los viejos guerreros, dexado el deseo de los otros capitanes en su ánimo, anteposieron para sí aqueste, y pensaron que le deviessen mayormente eligir por su capitán. 3 Porque todas cosas que pareçían deverse cobdiciar para ser en lo avenidero grand capitán, abondosamente las vían en  {\persName Hanníbal}. Tenía prompto el consejo para las singulares y loables fazañas, y al consejo no le faltava industria nin osadía; 4 y al varón de ánimo invincible no espantavan peligros nin algunos trabajos del cuerpo, que suelen retardar a todos los otros y retraerlos de las cosas que fazer se devan velar, apressurar y fazer todas las cosas que parecían dignas de valiente guerrero o de singular capitán. 5 Usando  {\persName Hanníbal} d'estas artes, militó tres años so capitanía de  {\persName Hasdrúbal}, en el qual tiempo, de tal manera pudo aquistar el amigable favor de todo el exército, que luego, después de la muerte de  {\persName Hasdrúbal}, con muy grand consentimiento de los guerreros fue declarado capitán del exército {\small\itshape [3P. omite quam praerogatiuam ... controuersia comprobauit ... [3.1.] Sex et uiginti ... exercitu declaratus.]} .
\pend
\pstart
3.1 Tengo sabido aquesto del nascimiento de  {\persName Hanníbal}: % image:../public/images/1491/166r.png
[166r,a] Sábese que era de nueve años quando el padre le aduxo en  {\placeName España}, y desde aquel tiempo fasta la muerte de  {\persName Hasdrúbal}, segund es auctor  {\persName Polibio}, se cuentan dies y siete años. 2 Y desd'el día que  {\persName Hanníbal} reçibió el exército y la administración de toda la cosa pública, no interpuso luenga tardança a todo su pensar en començar guerra contra el pueblo romano, segund que ya mucho antes lo tenía propuesto en su ánimo. 3 Ca primero estava indignado contra los romanos quasi común con todos los  {\name carthagineses}, porque avían perdido a  {\placeName Sicilia} y a  {\placeName Sardinia}. Juntávase con esto la enemistad particular y quasi hereditaria reçebida de  {\persName Hamílcar} su padre, que ningún otro de los capitanes  {\name carthagineses} fuera más enemigo al nombre romano. 4 Es puesto en memoria que sacrificando  {\persName Hamílcar} en el tiempo que él aparejava para passar en  {\placeName España} estriñió con firme juramento a  {\persName Hanníbal}, que aún era mochacho, para que, siendo de edad líçita de lo emprender, luego se mostrasse enemigo del pueblo romano. 5 Y la memoria d'esto y áspera recordación, como una ymagen de la enemistad del padre se ponía ante los ojos del mançebo y pungía su ánimo para aparejar siempre lo que pertenecía a la destruyción del Imperio romano. 6 Allende d'esto, el vando barchino no cansava de le incitar a aguijar de continuo que aquistasse grand poderío y favores con las armas y capitanía. 7 Aquestas causas, assí públicas como particulares, que exhortavan a  {\persName Hanníbal} para emprender guerra con los romanos, fizieron qu'el mançebo feroce tomasse occasión de innovar los negocios de la tal materia.
\pend
\pstart
4.1 Eran en  {\placeName España} por aquel tiempo los  {\name saguntinos} quasi medianos dexados en libertad por derecho % image:../public/images/1491/166r.png
[166r,b] de la pleytesía entre los términos de los romanos y de los  {\name carthagineses}. Y estos  {\name saguntinos} después se llegaron en todo a los romanos y refirmada la compañía eran avidos por muy leales honradores del imperio romano. 2 Por ende  {\persName Hanníbal} propuso en su ánimo que no podría fazerse otra cosa más oportuna para incitar la yra de los romanos y el fuego que él mucho deseava encender, que affligir a los  {\name saguntinos} con armas y con guerra. 3 Mas antes que de manifiesto fuesse contra los compañeros del pueblo romano, determinó yr con la hueste contra los  {\name olcadas} y los otros allende de  {\placeName Hebro}. Y compelidos a se le dar, fallaría alguna causa de dañar a  {\placeName Sagunto} , tal que podiesse pareçer fazerse no tanto por acuerdo primero pensado, como porque los  {\name saguntinos} travassen la guerra. 4 Assí que, ya primero compelida la gente de los  {\name olcadas} que se le diesse, luego fue \edtext{contra}{\Afootnote{; conra ()}} los  {\name vasceos} y  {\name taloles} los campos y tomoles por fuerça muchos pueblos y tomó assí mesmo a  {\placeName Hermandica} y a  {\placeName Arboloca}, çibdades muy ricas; y posiera debaxo de su poderío ya quasi toda la comarca, 5 quando muchos fuydizos de  {\placeName Hermandica} unos a otros entre sí se confirmaron y fizieron conjuración contra los  {\name carthagineses}, y costriñieron a ello toda la muchedumbre y llegaron a su compañía los desterrados de los olcadas; 6 y desdende, con exhortaciones fizieron que los carpentanos que comarcavan con ellos de común consejo salteassen a  {\persName Hanníbal}, en bolviendo, sin él pensar tal cosa. 7 De ligero los pueblos, cobdiciosos de guerra y lastimados de rezientes injuros acordados en esta sentencia, tomaron armas y, juntada muy grand muchedumbre (ca sobre çient mill eran los que se juntaron), salieron çerca del río % image:../public/images/1491/166v.png
[166v,a] Tago contra  {\persName Hanníbal}, que tornava de los vasceos; 8 y en viéndolos assí de súbito, detoviéronse las señas de los  {\name carthagineses} y ovo en toda la hueste encaminada pavor y rebuelta. Y non se dubda que aquel día los  {\name carthagineses} reçebieran muy grand rompimiento, si, assí conmovidos a temor por la venida rebatada de los enemigos y cargados de robo, venieran a las manos con gentes muy feroces. 9 Pero el próvido capitán  {\persName Hanníbal}, considerando estas cosas, acordó sobreseer en la batalla, y puso allí su real. Y en la noche seguiente, con el mayor silencio que él pudo, fizo que sus compañas passassen el río, y dexó vazío de toda defensa el logar çercano por donde podían passar ligeramente los enemigos, en parte fingiendo que avía miedo, y en parte porque, dada aquella occasión, podiesse atraer a los bárbaros que passassen el río. 10 Nin la opinión y acuerdo para engañar al enemigo salió baldía al varón, que sin debate fue ventajoso en astucia a todos los capitanes de su edad. Porque los bárbaros naturalmente feroces y que confiavan mucho allende de lo que les cumplía de su muchedumbre, pensando que los  {\name carthagineses} yvan espantados, a grand priessa se lançaron al agua. 11 Y los  {\name carthagineses}, primero con la gente de cavallo suya y luego con toda la hueste, dieron en ellos desordenados y rebueltos antes que del todo oviessen passado el río; y después de muerto grand número d'ellos fizieron fuyr a todos los otros.
\pend
\pstart
5.1 Después d'esta victoria todos los pueblos allende Hebro se dieron a los  {\name carthagineses} sino los  {\name saguntinos}, los quales aunque vían cómo  {\persName Hanníbal} les estava sobre las çervizes, mas, teniendo confiança en la compañía del pueblo romano, pensaron que se podrían defender, y embiaron a grand priessa sus embaxadores % image:../public/images/1491/166v.png
[166v,b] a  {\placeName Roma} que fiziessen saber al senado en \edtext{quánto}{\Afootnote{; quando ()}} grand peligro estavan sus cosas y para que pediessen ayuda contra el muy áspero enemigo, que ya sin dubda les venía a fazer fuerça. 2 Escassamente los embaxadores de los  {\name saguntinos} eran partidos de  {\placeName España} para yr a  {\placeName Roma}, quando  {\persName Hanníbal}, ya de manifiesto para romper guerra, vino sobre  {\placeName Sagunto} con todas las compañas a çercar la çibdad y la combatir. 3 Aquesto sabido en  {\placeName Roma}, discutieron en el senado sobre la injuria de los compañeros, y los padres determinaron con acuerdo de mansedumbre embiar a  {\persName Publio Valerio Flacco} y a  {\persName Quinto Fabio Pamphilo} que fuessen a  {\persName Hanníbal} a que se bolviesse de sobre  {\placeName Sagunto} y que, si no obedeciesse a este requirimiento, mandáronles yr a  {\placeName Carthago} y que por vigor de la pleytesía demandassen al mesmo capitán que le avía corrompido.
\pend
\pstart
6.1 Cuenta  {\persName Polybio} que  {\persName Hanníbal}, oýdos aquestos embaxadores, les dio vanas respuestas.  {\persName  {\persName Livio}}, al contrario, escrive que nin fueron oýdos nin reçebidos en el real. Mas ambos concuerdan en esto diziendo como venieron en  {\placeName España} y desdende passaron en  {\placeName África} y llegaron a  {\placeName Carthago} y exposieron la mensajería de los padres, y dizen que fallaron tan contrario el vando Barchino, que, menospreciados y desechados, se bolvieron a  {\placeName Roma} sin fazer algo de aquello por qué yvan.
\pend
\pstart
7.1 Avía, en todo, entre los del senado de los  {\name carthagineses}, dos vandos, de los quales el uno ovo comienço desde  {\persName Hamílcar}, renombrado Barcha, que como por heredad veniera al fijo  {\persName Hanníbal}, y de nuevo avía cresçido aquel vando en tantos favores que, assí en casa como fuera de la çibdad, en todas las cosas que se avían de determinar se avía por soberano. 2 Era el principal del otro vando Hanno, varón grave y de grand dignidad % image:../public/images/1491/167r.png
[167r,a] en la república, pero más egregio o señalado en la toga que en armas. 3 Aqueste, quando los embaxadores romanos venieron a  {\placeName Cartago} a querellarse de la injuria de los compañeros, dizen que fue solo él, que, repugnando quasi todo el senado a sus dichos, juzgava que la pleytesía se devía guardar al pueblo romano y les amonestava que se guardassen de la guerra que traería extrema destruyción a su patria. 4 Y si en esto los  {\name carthagineses} quisieran mirar no solamente el ánimo de  {\persName Hannón}, quanto considerar su consejo, y antes lo quisieran oýr por auctor de la paz que de la guerra, ovieran dado más sancto consejo a sý mesmos y a su cosa pública. 5 Mas seguiendo ellos para tantos males el furor y cobdiçia de aquel mançebo, que después redundaron en sus cabeças, dieron a ello occasión desde entonçes. 6 Por ende conviene de razón a los varones sabios y muy buenos governadores de la república que en las cosas humanas miren más el fin qu'el comienço, y que, primero que vengan a las armas o a la rotura de las guerras, experimenten guiar todas las cosas por consejo.
\pend
\pstart
8.1 Los  {\name saguntinos} en el mesmo tiempo que  {\persName Hanníbal} les puso çerco, viendo que la guerra se les fazía contra derecho y contra el dever, sostuvieron muchos meses con ánimos endurezidos y enemigables el çerco. 2 A la postre, sobrepujando el enemigo en muchedumbre de gente (ca se cree tener allí en armas çiento y çinquenta mill ombres), quando ya en la mayor parte los muros eran derribados, quesieron más atender la destruyçión de la çibdad que venir en poder del muy áspero enemigo. 3 Hay auctores que dizen ser tomada  {\placeName Sagunto} a ocho meses después que fue començada a combatir. Y a esto no pareçe que  {\persName Livio} consienta, nin que assigne otro çierto tiempo de aquel çerco en medio. 4 Pero la toma forçosa d'esta muy rica çibdad fue en muchas cosas en el comienço grande ayuda a la empresa de  {\persName Hanníbal}, ca pudo retener en fe algunas çibdades que él temía aver de faltar al imperio de los  {\name carthagineses}, por el exemplo del perdimiento de  {\placeName Sagunto} ; 5 Y pudo confirmar los ánimos de los guerreros enriqueçido el exército con muy grande robo que allí ovieron, y él embió a  {\placeName Carthago} joyas preciosas de los despojos {\small\itshape [4joyas: traducción del lat. \textit{munera}.]}  de los  {\name saguntinos} con que pudo fazer que los principales çibdadanos le fuessen obligados y esto viessen prestos para la guerra que entendía fazer contra el pueblo romano, no solamente en  {\placeName España}, segund que muchos entonçes pensavan, mas aún en Italia como él lo tenía en voluntad.
\pend
\pstart
9.1 En tanto, venidos a  {\placeName Roma}, los embaxadores romanos, recontaron las vanas respuestas que les dieran y quasi en el mesmo tiempo llegó la nueva de la destruyçción de  {\placeName Sagunto} ; y el pueblo romano començó tarde arrepentirse por no aver socorrido en tan extremo peligro a la república de los compañeros que les demandavan ayuda. 2 Assí que la plebe y los padres, conmovidos por misericordia increýble y juntamente encendidos en saña, mandaron que los cónsules por suerte reçebiessen las provinçias. Cupo por suerte  {\placeName España} a Publio Cornelio, y  {\placeName África} a Tito Sempronio, a quien cupo  {\placeName Sicilia}; Estos dos eran entonçe cónsules {\small\itshape [5a quien ... cónsules: una traducción más exacta hubiera sido 'Estos, de hecho, eran los cónsules, cuando [Sempronio] pasó a Sicilia'.]}  3 y embiaron por embaxadores a  {\placeName Carthago} varones principales de la çibdad, los quales después de luenga disputaçión del derecho de la pleytesía, quando ya ovieron derramado semientes de la contienda avenidera, quasi en el acatamiento de todos y con grand rigor ofreciessen la guerra; respondieron los  {\name carthagineses} no con menor osadía que ellos la acceptavan, pero con mal consejo, segund qu'el mesmo negoçio y el fin de la guerra después lo declaró.
\pend
\pstart
10.1 Y  {\persName Hanníbal}, quando ovo conosçido todo lo que en el senado de los  {\name carthagineses} era negociado, pensado ser ya tiempo que segund desd'el comienço tenía propuesto, fuesse a Ytalia; con grande estudio aparejava las cosas necessarias, fizo llegar flota, començó a reçebir consigo ayudas de las çibdades muy leales y mandó que todas las compañas se juntassen en  {\placeName Carthagine la Nueva}. 2 Después d'esto fue él a Cádiz, y ante todas cosas determinó ser cosa muy provechosa guarneçer a  {\placeName África} y a  {\placeName España} de oportunas guardas, porque, mientra él yva en Ytalia, no dexasse a los romanos aquellas tierras vazías de toda guarnición. 3 Assí que primero embió en  {\placeName África} mill y dozientos de cavallo y treze mill peones de los de  {\placeName España}. Allende d'esto fizo juntar a su llamamiento quatro mill peones de  {\placeName África} para el amparo de  {\placeName Carthago} y quiso juntamente aver rehenes y guerreros. 4 Prefirió a  {\placeName España} a su hermano  {\persName Hasdrúbal} y dexole armada por mar de cinquenta naves y fasta dos mill de cavallo y fasta doze mill peones. 5 Y con estas guarniciones refirmó ambas provinçias, no porque pensasse ser bastantes aquellas cosas contra el poderío de los romanos si la difficultad y peso de la guerra se oviesse de sostener en  {\placeName España} o en  {\placeName África}, mas porque cuydava ser assaz para arredrar de los términos y fronteras a los enemigos, mientra que él por el camino de la tierra yva a passar y fazer guerra en Ytalia.
\pend
\pstart
11.1 Otrosí tenía sabido qu’el pueblo carthaginés, segund la grandeza de su imperio, ligeramente podía quando quisiesse aparejar nuevas compañas, y, no sólo rehazer en casa el exército si menester fuesse, mas aun embiar en Ytalia gente % image:../public/images/1491/167v.png
[167v,b] fresca para suplir lo que faltasse, 2 pues que avían echado de sobre sus cervizes guerra tan grave y tan peligrosa como les fiziera la saña de los soldados, y primero siendo capitán  {\persName Hamílcar}, y desdende  {\persName Hasdrúbal}, y a la postre capitaneando  {\persName Hanníbal}, uno tras otro siempre vencedores, 3 de tal manera avían acrescentado el imperio púnico o carthaginés, que en el tiempo quando  {\persName Hanníbal} vino en  {\placeName Ytalia} sus favores y poderío a la luenga y a la larga estavan extendidos. 4 Ca toda la costa o ribera de  {\placeName África} junta a nuestro mar desde las aras de los philenos, no lexanas de la grand Syrte, fasta las columnas de Hércules, donde hay longura de dos mill passos en el estrecho; y, passado aquel mar tan angosto que passa entre  {\placeName África} y  {\placeName Europa}, tenían los  {\name carthagineses} quasi toda  {\placeName España} fasta el  {\placeName monte Pyreneo} que divide aquella provincia de la provincia de  {\placeName Galia}.
\pend
\pstart
12.1 Aquestas cosas d’esta manera dispuestas en  {\placeName España} y en  {\placeName África},  {\persName Hanníbal} bolviose a  {\placeName Carthago la Nueva}, do ya estavan las compañas puestas en orden y prestas en armas. 2 Assí que, viendo que no devía más atender, llamada la gente a oyr su razonamiento, quiso confirmar con buenas palabras los ánimos de los guerreros proponiéndoles sperança de grandes cosas, y, predicándoles la fertilidad de  {\placeName Ytalia}, fízoles memoria de la amistad de los galos y afincadamente les amonestó que con alegres ánimos emprendiessen la yda.
\pend
\pstart
13.1 Otro día seguiente, moviendo de  {\placeName Carthagine} junto a la costa del mar {\small\itshape [6la costa del mar: interpretación correcta del término lat. erróneo \textit{horam}.]} , fue fasta el  {\placeName Hebro}. Hay quien diga que en la noche luego çercana apareçió a  {\persName Hanníbal} por visión en sueños un mançebo de maravillosa fermosura que le exhortó primero le seguiesse en la yda para  {\placeName Ytalia} como a guía. 2 Y luego después le apareçió una serpiente con grande estruendo y tan grande que se fallarían pocas de tanta grandeza, % image:../public/images/1491/168r.png
[168r,a] y, deseando saber qué significasse aquella sierpe, pareçiole oýr que era la destruyción de Ytalia. 3 Ni es de maravillar que correspondiessen los intentos cuydados y pensamientos vehementes a las cosas que él, velando, pensava siempre para la guerra de Ytalia, y que dende proçediessen algunas species que de noche parían los tales pensamientos, que ante sí traýan algunas vezes semejança de victoria, y otras vezes de matança y de incendio y de otras tribulaciones de la guerra. 4 Algunas vezes, segund dize  {\persName el Orador}, se faze que nuestros sermones y pensamientos paren alguna cosa en el sueño tal, como  {\persName Ennio} escrivió de  {\persName Homero} que él muy muchas vezes mientra velava solía pensar y fablar.
\pend
\pstart
14.1  {\persName Hanníbal} ya passadas las alturas del  {\placeName Pyreneo}, atraýdos a su amistad con muchos dones los ánimos de los galos, llegó en pocos días al  {\placeName Rhódano}. 2 El  {\placeName Rhódano} es río grande {\small\itshape [7río grande: traducción del lat. \textit{amnis}.]}  que comiença levantarse non lexos de las fuentes del  {\placeName Istro} o  {\placeName Danubio} y del  {\placeName Rheno}; y dende, quasi a ochoçientos estadios, se mete en el  {\placeName lago Lemano} y sale d’él faza el occidente y, bolviendo, passa por las  {\placeName Galias}, y assí, por entrar \edtext{en el río}{\Afootnote{; enel el rio ()}}  {\placeName Araris} como por se le llegar otros ríos, creçe mucho en aguas; al cabo, entre los volcas y cavares, por muchas cabeças entra en el mar. 3 La gente de los volcas en aquel tiempo morava çerca de ambas riberas del Rhódano y era muy abundante de ombres y muy rica entre las gentes gálicas. 4 Aquestos en la primera venida de  {\persName Hanníbal} passaron el río y detoviéronse ende en la ribera por vedar el passaje a los  {\name carthagineses}. Ca, apaziguados ya todos los otros galos, no pudo  {\persName Hanníbal} compelir a éstos con miedo ni aquistarlos con dádivas para que quesiessen más provar la amistad de los  {\name carthagineses} que la fuerça. % image:../public/images/1491/168r.png
[168r,b] 5 Assí que él, cuydando que con el tal enemigo se devía aver engañosamente, mandó a  {\persName Hannón}, fijo de  {\persName Bomílcar}, que passasse el Rhódano escondidamente y con una parte de \edtext{las}{\Afootnote{; los ()}} compañas acometa a los galos mientra que d’esto estavan descuydados. 6  {\persName Hannón}, segund le fue mandado, yendo rodeando, passó el río por dónde le pareçió mejor passaje, y dio buelta con sus compañas y llegó primero al real de los enemigos que le podiessen ellos veer o conosçer lo que se fazía. 7 Los galos, oyendo grita a las espaldas, como no tenían espaçio de aver consejo, nin de tomar las armas, y ya  {\persName Hanníbal} estava a la fruente con muchos navíos aparejados para passar, de súbito se partieron del real y, quanto más podieron yr corriendo, començaron a fuyr. {\small\itshape [8P. omite el lat. \textit{in praecipitem}.]}  8 D’esta manera, echados los enemigos de la ribera contraria, passaron el río en salvo todas las otras compañas de los  {\name carthagineses}.
\pend
\pstart
15.1 En el medio tiempo venieron espessas nuevas a  {\persName Publio Cornelio Scipión} de la venida de  {\persName Hanníbal}. El qual cónsul era poco antes llegado a Marsella y, por lo saber más de çierto, embió una ala de muy escogidos cavalleros a descobrir los acuerdos de los enemigos y avíales mandado yr presto, y assí lo fizieron, y recontraron con quinientos cavalleros númidas que tanbién embiava  {\persName Hanníbal} a mirar en qué logar estava puesto el real de los romanos. 2 Ninguna tardança se interpuso en començar la pelea: los unos y los otros pelearon agramente; al fin, los romanos levaron ventaja después de muertos muchos de los suyos y muchos más de los enemigos, que fueron a la postre fuyendo.
\pend
\pstart
16.1 Passado aquesto,  {\persName Hanníbal}, ya çertificado en qué logar estavan las compañas de los romanos, començó pensar consigo si fuesse mejor % image:../public/images/1491/168v.png
[168v,a] continuar su camino para yr en Ytalia o levar su hueste contra el cónsul que ende era çercano, y experimentar lo que succediesse en el negocio. 2 Occurriéndole de cada parte razones, al fin, siendo su ánimo dubdoso, los embaxadores de los  {\name Boyos} le induxeron en aquella una parte que, todas cosas dexadas, passasse en  {\placeName Ytalia}. 3 Porque ya los  {\name Boyos}, ante que  {\persName Hanníbal} passasse el  {\placeName Pyreneo}, avían preso por engaño los embaxadores de los romanos y, siendo  {\persName Manlio} pretor opprimido con grand pérdida de su gente, fueran solicitados los insubres de se passar al favor de los  {\name carthagineses}, siendo d’esta causa indignados que los romanos poco antes oviessen embiado colonias de pobladores romanos a  {\placeName Placencia} y a  {\placeName Cremona}.
\pend
\pstart
17.1 Assí que  {\persName Hanníbal}, conmovido con los consejos d’estos, movió su real y, yendo por la ribera del  {\placeName Rhódano} el río arriba, en pocos días llegó al logar que los  {\name galos} llaman la  {\placeName Ysla}, que fazen allí los dos ríos Araris y Rhódano, que corren de diversos montes fasta llegar a Lugdunio, muy honrada çibdad de  {\placeName Galia}, la qual, segund sopimos, fue después, passados luengos tiempos, allí fundada por  {\persName Planco Numacio}. 2 Desde allí fue caminando  {\persName Hanníbal}, y entró en tierra de los allobroges o del dalphinadgo y, atajadas y abenidas las discordias de dos hermanos que contendían entre sí sobre el reyno, vino al río Druencia por las tierras de los castinos y voconcios. 3 El río  {\placeName Druencia o Durença} nasçe de los  {\placeName Alpes} y, descendiendo con rebatada corrida, se lança en el  {\placeName Rhódano} y, porque muchas vezes muda los vados, apenas se puede passar a pie. Pero ya passado el río por logares más descobiertos, quanto la facultad lo consentió, fizo encaminar su exército fasta los  {\placeName Alpes}.
\pend
\pstart
18.1 Dizen que para subir a los  {\placeName Alpes} fueron tantos y tan dañosos los trabajos que  {\persName Hanníbal} % image:../public/images/1491/168v.png
[168v,b] ende ovo, que algunos auctores de los que ende se fallaron en tiempo de aquella guerra, escriven que ellos oyeron dezir a  {\persName Hanníbal} aver él perdido en passar los  {\placeName Alpes} sobre treynta mill ombres y muy gran número de bestias. 2 Y no sólamente con los moradores de las montañas le fue neçessario lidiar muchas vezes, mas aun ovo de trabajar tanto contra las estrechuras y asperezas de los caminos, que en algunos logares le fue menester que abriesse el camino podreciendo los grandes berruecos con fuego y con vinagre. 3 Al día quinto décimo, ya passados los  {\placeName Alpes}, descendió en el campo Taurino. Y por esto me pareçe verisimile que  {\persName Hanníbal} con los  {\name carthagineses} passasse por el  {\placeName monte Genevo} qu’el vulgo assí lo llama, el qual de la una parte derrueca de sý al  {\placeName río Druençia} y de la otra da salida para yr a  {\placeName Turín}.
\pend
\pstart
19.1 Después que vino en  {\placeName Ytalia}  {\persName Hanníbal}, es diffícile dar fe affirmando el número de quántas compañas se fallaron allí con él, segund la diversidad de los auctores que d’esto escrivieron. 2 Hay algunos que dizen aver passado con él çient mill peones y veynte mill de cavallo, otros que veynte mill peones y seys mill de cavallo africanos y españoles. Otros escriven que con los galos, y añadidos los lígures, se fallaron con él ochenta mill peones y diez mill de cavallo. 3 Mas devemos creer que no podiesse allí llegar con tan grand número de compañas como los primeros affirman, después de passadas tantas tierras y reçebidos tantos daños y quiebras. Ni es de creer que fuesse el número tan pequeño como quieren los segundos, maiormente occurriendo a la memoria las cosas que después fizo. 4 Aquestos que siguen la vía mediana pareçe que digan lo más probable, pues que de setenta mill peones y de diez mill de cavallo que él movió % image:../public/images/1491/169r.png
[169r,a] de  {\placeName España}, aduxo la mayor parte en Ytalia y se sabe por çierto que recurrió a él muy grand muchedumbre de lígures y de galos que en aquel tiempo no eran menos encendidos en enemistad contra los  {\name Romanos} que los  {\name carthagineses}.
\pend
\pstart
20.1 Venido ya  {\persName Hanníbal} desde los Taurinos en el campo de los insubres, tovo ende contra sí a Publio Cornelio Scipión, el qual de muy grand priesa bolvió desde Marsella en Ytalia y, passado el Pado y el Ticino, se aposentó no lexos del enemigo; y, aviendo entre medias pequeño espaçio, ambos los capitanes fueron a mirar el uno el real del otro y el otro del otro, y escaramuçaron a cavallo, y como en la pelea algund tanto de tiempo pareçieron eguales. 2 Al fin los romanos, por ser ferido el cónsul, y los cavalleros númidas poco a poco los rodeavan a las espaldas, fueron constriñidos retraerse passo a passo amparando al cónsul fasta que se recogieron en el real. 3 Hay quien escriva que su fijo, renombrado después Africano, que entonçe començava tener barba, \edtext{conservó}{\Afootnote{; conserno ()}} al padre Cornelio. La qual loança aunque sea muy grande en tan reziente y tierna edad, pero no pareçe ajena de verdad, nin de las muy claras fazañas que después puso en obra. 4 Scipión pudo en esta pelea conosçer por experiençia quánto el enemigo le tenía ventaja en gente de cavallo. Por ende, determinó yrse a aquellos logares en que sus peones estoviessen más seguros y peleassen más a su provecho. 5 Assí que la noche seguiente, con el mayor silençio que pudo, fue passar el Pado con todo el exérçito y fuele aposentar en el campo Plaçentino. Poco después llegó ende \edtext{Tiberio}{\Afootnote{; Tibero ()}} Sempronio Luengo, llamado por letras del senado desde  {\placeName Sicilia} para que ambos cónsules de común capitanía y de común consejo administrassen la cosa pública.
\pend
\pstart
21.1 Otrosí % image:../public/images/1491/169r.png
[169r,b]  {\persName Hanníbal}, después de la pelea que se fiziera entre cavalleros con todas sus compañas, seguió a Scipión fasta el río Trebia y aposentose ende çerca, esperando que por la çercanía de los reales avría alguna facultad de pelear, que él muy mucho deseava que se le ofreçiesse. 2 Non sólamente porque él no podía allí mucho detenerse, segund la mengua de los mantenimientos, mas aun porque temía la poca firmeza de los galos, los quales, seguiendo la esperança de las novedades y la fama de la victoria de los cavalleros, 3 creýa que, segund de ligero venieran en su fe y amistad, assí con liviano momento podría ser que, si la guerra durasse en sus tierras, mudarían la enemistad de los romanos contra el que era auctor de la guerra. Por ende, en todas maneras buscava materia donde nasçiesse occasión de cometer batalla.
\pend
\pstart
22.1 En los mesmos días oportunamente acaesçió qu’el cónsul Sempronio, fallada por los campos esparzida gente de los enemigos cargada de robo, dio en ellos y fízolos fuyr; y d’esto, que se fizo prósperamente, fizo conjectura del fin de todo el negoçio, con esperança que, si en batalla veniessen a las manos, conseguiría victoria. 2 Y con el tal intento, antes que Scipión convaleçiesse o nuevos cónsules fuessen criados, cobdiçioso de obrar alguna muy clara fazaña, determinó de pelear, comoquier que el collega o compañero lo tachava y dava bozes diziendo que ninguna cosa se podía fazer más fuera de tiempo que poner en peligro la suma y estado de la república teniendo allí por contrarios quasi todos los galos. 3 Las differençias d’estos cónsules por espías se \edtext{denunciavan}{\Afootnote{; dennnciavan ()}} a  {\persName Hanníbal}, y, sabidas, luego el cauteloso capitán mandó a su hermano Magón que se posiesse con escogida compaña en un logar que él avía fallado entre ambos % image:../public/images/1491/169v.png
[169v,a] los reales, vestido de todas partes de çarçales y de spinos. 4 Y dio cargo a los cavalleros númidas que arremetiessen al real de los romanos y açomassen al enemigo para le enridar a batalla. 5 En tanto, él fue sacando su gente y puso la az en orden y fizo qu’el exérçito comiesse y beviesse, porque, si qualquier occasión recresçiesse de venir a las manos, estoviesse presto y aparejado en armas. 6 El cónsul Sempronio, al primer rebato de los númidas, súbitamente fizo salir la gente de cavallo y luego tras ellos seys mill peones, y a la postre fizo salir del real todas las otras compañas.
\pend
\pstart
23.1 Era tiempo de eladas y la ynvernada era súmamente áspera y mucho más en aquellos logares que contornan los Alpes y el Apenino. Los númidas, segund les fue mandado, fueron sosacando poco a poco los romanos aquende del río Trebia, y quando llegaron al logar donde podieron veer las señas de los suyos súbitamente, bolvieron los cavallos contra los enemigos que venían esparzidos. 2 Tienen en costumbre los númidas de se retraher de su voluntad muchas vezes y de se retener luego quando les pareçe complidero, y arremeten de nuevo al enemigo con mayor ímpeto que primero. 3 Sempronio prestamente fizo tornar su gente de cavallo a los socorros de las otras sus compañas y puso en orden la az, segund la neçessidad del tiempo, para cometer la batalla con el enemigo ya antes aparejado a la venidera lid. 4 Estava ya ende presente con sus compañas ordenadas  {\persName Hanníbal} para gozar de la presente occasión de la contienda, la qual se començó \edtext{primero}{\Afootnote{; prmero ()}} entre los de ligera armadura y después entre los de cavallo, pero los cavalleros romanos, no podiendo sofrir el embate de los enemigos, de ligero se retraxeron. 5 Y desdende, % image:../public/images/1491/169v.png
[169v,b] las legiones arremetieron con tan grand contienda y con tan aparejados ánimos para pelear, que mostraron poder resistir si ovieran de lidiar con los otros peones sólos. Mas de la una parte ponían espanto los elefantes {\small\itshape [9P. omite el sujeto lat. \textit{equites}.]}  y de la otra apretavan los peones agramente contra los cuerpos que para pelear estavan atormentados de fambre y de frío. 6 De manera que los romanos sofrieron más aquesta difficultad y peso de los daños çircunstantes con los ánimos que con las fuerças, y continuaron la pelea fasta que Magón, salido de la çelada con grita y con estruendo, arremetió contra los que d’esto no temieran y la az mediana de los  {\name carthagineses}, por mandado de  {\persName Hanníbal}, arremetió contra los çenomanos. 7 Entonçes, començando a fuyr los ayudadores, fueron quebrantados los ánimos de los romanos ca, segund se dize, fasta diez mill peones de los romanos rompieron por medio de los enemigos y se fueron a Plazençia, y de los que yvan fuyendo de las otras compañas, grand parte mataron los  {\name carthagineses}. 8 Con todo, pudo escapar el cónsul Sempronio de las manos de los enemigos con soberano peligro, y la victoria que ovieron los  {\name \edtext{carthagineses}{\Afootnote{; Carthagines ()}}} no fue sin sangre, porque perdieron muchos de los suyos y quasi todos los elefantes.
\pend
\pstart
24.1 Después d’esta batalla  {\persName Hanníbal}, yendo por toda la comarca, pudo talarlo todo con fierro y con fuego, y tomó algunos logares y rompió y fizo fuyr con poca gente gran muchedumbre de los moradores de aquella tierra que venían a pelear sin algund orden militar. 2 En la primavera, más presto de lo qu’el tiempo demandava, movió de los aposentamientos del ynvierno para passar en Toscana {\small\itshape [10Toscana: substitución del topónimo lat. \textit{Etruria}.]} , y llegado a las cumbres del Apenino, ya çerca del passaje, otra vez le fizo la tempestad % image:../public/images/1491/170r.png
[170r,a] tan fiera, que, repelido dende reduxiesse el exérçito en el campo de Placencia y, passado breve espaçio, costriñido de muchas causas y assaz neçessarias, de nuevo, segund \edtext{que}{\Afootnote{; qua ()}} antes instituyera, pensó tornar a seguir su passaje. 3 Ca poco menos ya rodeado de las asechanças de los galos, supo guardarse con cauteloso ingenio del peligro de su cabeça, porque los galos, aviendo a mal que tan luengamente la guerra se fiziesse en sus mesmos campos, buscavan manera de matar a  {\persName Hanníbal}, como a cabeçera sola de toda aquella guerra. Y él, complido d’este peligro, pensava apressurarse en passar el exérçito a la otra provinçia.
\pend
\pstart
25.1 Llegávase a esto, que para la opinión de las gentes y para la alegría de sus guerreros creýa ser muy \edtext{provechoso}{\Afootnote{; provochoso ()}} que las fuerças de los  {\name carthagineses} pareçiessen tan grandes y tan grande el ánimo del capitán, que osasse passar con la gente de armas en las tierras çercanas a Roma. 2 Por ende, todas cosas postpuestas, movió el real y fue passar las cumbres del Apenino por tierra de los lígures y aquella vía que va a las paludes {\small\itshape [11P. omite el lat. \textit{ac planiciem}.]}  del río de Arno y desçiende en Toscana en tiempo que por aquellos días el río estava muy cresçido y las aguas se esparzían por todos los campos d’enrededor de la ribera. 3 Assí que  {\persName Hanníbal}, levando consigo tan grande exérçito, antes que saliesse de los logares paludosos no pudo esquivar que no incurriesse grand pérdida de ombres y de bestias. 4 Y el mesmo capitán, subido en un elephante solo que avía quedado de muchos, yva puesto en alto, mas trabajando tantos días y tantas noches, con el destempramiento del ayre y el no dormir ovo de perder el uno de sus ojos.
\pend
\pstart
26.1 En tanto, Gayo Flamineo cónsul, tomado el exérçito de Sempronio, era venido a Arezo sin lo consultar % image:../public/images/1491/170r.png
[170r,b] con el senado, que d’esto ovo grande enojo, porque dexado en  {\placeName Roma} su colega o compañero Gneo Servilio, se fue ascondidamente a la provinçia sin insignias y sin lictores. 2 La plebe avía mucho loado a este varón naturalmente feroçe, y favoreçíale demasiadamente, y con esto le avía fecho tanto más osado, que bien pareçía aver él de fazer \edtext{todas}{\Afootnote{; todos ()}} las cosas desacordadamente y sin buen consejo. 3 Lo qual, luego que lo supo  {\persName Hanníbal}, determinó enridar el ánimo del cónsul y trabajar en todas maneras cómo, antes que el otro cónsul su compañero se juntasse con él, le atraxesse a pelear.
\pend
\pstart
27.1 Por ende, movido su real por el campo fessulano y fasta çerca de Areço, con grande ímpeto hinchió toda aquella tierra de espantos talando quanto fallava con fierro y con fuego, y no fizo fin a las talas y quemas antes que de poco en poco fue a talar los campos juntos a los montes de Cortona y desde allí vino al lago Transimeno. 2 Y allí miró bien el logar queriendo tomar al enemigo en assechanças, assí que colocó la gente de cavallo çerca de una vía angosta que fazen los que van al lago Transimeno baxo de unos oteros, y mandó que los de ligera armadura estoviessen tras los montes. 3 Y él, con todas las otras compañas, descendió a la llanura sospechando lo que acaesçió, qu’el cónsul romano no estaría quedo. 4 Ca los ombres furiosos de su natural de ligero caen en todas las trampas y assechanças de los enemigos y, las más vezes, despreçiados los consejos saludables, ponen en peligro los mayores negoçios del soberano estado.
\pend
\pstart
28.1 Flaminio sentiendo que se robavan y destruýan los campos de los compañeros, y los panes se talavan y se quemavan los edificios, súbitamente movió el real contra el acuerdo de todos los que juzgavan % image:../public/images/1491/170v.png
[170v,a] que devía atender al colega o compañero y dio priesa que la gente se encaminasse al enemigo. 2 Ya llegado, en poniéndose el sol, a las angosturas del lago de Transimeno, la gente, cansada del camino, estóvose en aquel logar la noche. 3 El día seguiente, començando amaneçer, sin descobrir nin proveer en cosa alguna, passó de la otra parte de la angostura. Y  {\persName Hanníbal}, que, todas cosas antes aparejadas, atendía occasión de fazer su negoçio, quando vio que el exérçito romano estava en el campo llano, dio a todos señal de arremeter. 4 Entonçe los  {\name carthagineses} de todas partes dieron en los enemigos çercados del lago y de los montes, y no sólamente los acometieron por la fruente, mas aun por las espaldas y por los lados. 5 Y al contrario, los romanos, sin guardar orden alguno militar, començaron a pelear estibados y amontonados en tiempo que la niebla les occupava la vista y lidiavan como en sombrío, de manera que fue maravilla con qué esfuerço tan luengamente sofrieron la contienda estando çercados de todas partes. 6 Ca sobre tres horas tan feroçemente pelearon, que muy grand terremoto non lo podieran oýr los que lidiavan, nin la az de los romanos fue primero desbaratada, que cundió por todos la fama como un cavallero nombrado Ducario avía muerto al cónsul, que discurría por toda la az como volando de unas partes a otras. 7 Entonçes los romanos, despojados del amparo del capitán y destituydos de toda esperança, bolvieron las espaldas, y unos se encaminaron a los montes y otros al lago, de los quales muchos fueron presos y muertos en la fuyda. De las compañas de los romanos cayeron en aquella batalla fasta quinze mill ombres y fasta diez mill escaparon dende por diversos caminos. 8 Otrosí dizen que % image:../public/images/1491/170v.png
[170v,b] fasta seys mill peones desd’el comienço de la pelea arremetieron a la estrechura del passo y se posieron en una altura de un otero, y que después, ya feneçida la batalla, los  {\name carthagineses}, dada fe que serían seguros, los engañaron y cometieron falsedad y los posieron en poder del vençedor.
\pend
\pstart
29.1  {\persName Hanníbal}, quando ovo conseguido aquesta tan grand victoria, mandó que liberalmente y sin rescate se dexassen muchos presos del nombre ytálico porque la fama de su humanidad y perdón se divulgasse por los pueblos, aunque su ingenio era muy ajeno d’estas virtudes. 2 Ca de su natural fue fiero e inhumano, y de tal manera fue disçiplinado desde su primera pueriçia, qu’él no avía aprendido leyes nin çiviles costumbres, mas guerras y muertes y enemigables trayçiones. 3 Assí que vino a ser muy cruel capitán y muy maliçioso en engañar a los ombres, y siempre puesto en cuydado de cómo podría engañar a su enemigo, y quando ya no podiesse por manifiesta pelea vençer, buscava engaños, segund de ligero pareçió en la presente batalla, y de la que antes cometió contra Sempronio çerca del río Trebia lo podemos judgar. Pero aquesto se diffiera para en otro logar.
\pend
\pstart
30.1 Denunçiado en  {\placeName Roma} que Flaminio era vençido y muerto con gran parte de sus compañas, a desora fue llena la çibdad de llantos, y unos lloravan el muy perdidoso daño público, otros el daño privado, otros ambas tribulaçiones. 2 Era cosa miserable veer la muchedumbre de las mugeres y de los varones que corrían a las puertas de la çibdad a demandar cada uno y çertificarse de sus parientes y amigos. 3 Fue puesto en memoria aquesto, que estando dos mugeres en muy gran cuyta y con ánimos pendientes, oyendo que sus fijos morieran, los vieron venir a desora y súbitamente expiraron con el grand % image:../public/images/1491/171r.png
[171r,a] plazer que d’ello ovieron.
\pend
\pstart
31.1 En el mesmo tiempo, embiava el cónsul Servilio quatro mill de cavallo a su compañero Flaminio antes que sopiesse de la batalla acaesçida en Transimeno. 2 Aquestos cavalleros, oýda en el camino la pérdida de los suyos, aviéndose recogido en Umbría, los cavalleros de los enemigos los saltearon y tomaron y traxeron presos a  {\persName Hanníbal}. 3 Avidos tantos daños y soberanas pérdidas de maneras diversas, que ya la cosa de los romanos estava en extremo peligro, plogoles buscar capitanía extraordinaria y que se criasse dictador. El qual remedio era el que acostumbravan poner a la república en los tiempos muy alterados. 4 Mas segund que todas las vías estavan çercadas de enemigos y el cónsul Servilio no podía venir a Roma, por nuevo exemplo el pueblo romano nombró dictador a Quinto Fabio, que después fue renombrado Máximo y maestro de la cavallería a Quinto Minucio.
\pend
\pstart
32.1 Era Fabio varón de grand consejo y prudencia y de soberana dignidad en la república. Assí que en aquel tiempo todos los çibdadanos miravan a este uno y se tenían por dicho que, siendo él capitán y ningún otro, se podía defender el estado de la çibdad. 2 Y él, sabiendo que assí fuese, aparejadas con gran cuydado las cosas neçessarias, salió de  {\placeName Roma} y, reçebido del cónsul Servilio el exército, y añadidas dos legiones, fue contra el enemigo.
\pend
\pstart
33.1 Ya  {\persName Hanníbal} desd’el lago Transimeno era ydo a Spoleto por camino derecho, y en la primera acometida tentó si podría tomar la çibdad; concurrieron los del logar a defender los muros tan esforçadamente, que  {\persName Hanníbal}, talados todos los campos Spoletanos y quemados los villajes y los edifiçios, fizo su passaje al Piceno y dende por los Marsos y Pelignos fue a Apulea.
\pend
\pstart
34.1 El dictador, seguiendo % image:../public/images/1491/171r.png
[171r,b] al enemigo, puso su real çerca de Arpos no lexos del aposentamiento de los  {\name carthagineses} con este acuerdo, que deteniéndose y con tardança se alongasse la guerra, pues que la demasiada feroçidad de los capitanes antepassados traxera en aquellos términos el estado romano y que no ser él vençido del enemigo tantas vezes vençedor era vençer. Assí qu’el capitán mudado, todo lo pudo mudar a desora. 2 Ca  {\persName Hanníbal} primero puso su gente en az para pelear y, quando vio que Fabio estava quedo, taló toda la comarca, porque, destruyendo los campos de los compañeros a ojo de los enemigos, enridasse al dictador a emprender batalla. Nin por aquesto se movió más Fabio y tenía los suyos quedos en su real.
\pend
\pstart
35.1  {\persName Hanníbal} avía pesar d’esta tardança del capitán romano y determinó mudar espessas vezes su real porque, yendo de unos logares en otros, podiesse nasçer alguna occasión de engañar al enemigo o de cometer batalla. 2 Por ende, passado el Appenino desd’el campo Arpinate, vino en Samio, y poco después –por aver dexado francamente algunos Capuanos que prendiera en Transimeno que le davan esperança de le dar a Capua–, movió su real; amonestado uno que sabía la tierra que le guiasse para levar el exérçito en el campo Cassinate. 3 Y el que guiava, engañado de la semejança del nombre de Cassilino por Cassino, por muy diversa vía levó a  {\persName Hanníbal} por el campo Calentino y Caleno fasta el campo Stellate. Y allí, quando vio çercada aquella comarca de todas partes de montes y de ríos, conosçido el error, mandó  {\persName Hanníbal} atormentar y matar cruelmente al que guiava. 4 Y, en tanto, usando Fabio de increýble paciencia, dexó al Carthaginés discurrir fasta que, occupado el monte Galicano y Casilino, pudo poner y refirmar guarniçiones % image:../public/images/1491/171v.png
[171v,a] de gente en los logares oportunos, de manera qu’el exérçito carthaginés o púnico, quasi atajado, pareçió ser costriñido a pereçer por mengua de mantenimientos o a fuyr feamente por escapar la vida; 5 y assí fuera, sino que  {\persName Hanníbal} con agudo engaño, pensado para lo que le cumplía, mandó a los suyos que, de lo robado por los campos y tenían abondosamente en el real, le traxessen fasta dos mill bueyes y mandó que les atassen manojos de leña seca a los cuernos y escogió varones ydóneos que en el comienço de la noche ençendiessen los hazes menudos que tenían atados los bueyes en los cuernos y los aguijassen fasta la subida de los montes. 6 Ninguna cosa se dexó de fazer de lo mandado; agujados los bueyes, començaron sobir los montes con sus manojos ençendidos, y las compañas poco a poco los seguían. Y los romanos, que mucho antes tenían occupados los passos en lo alto con guarniçiones de gente, espantados de la novedad y pensando que fuessen assechanças, súbitamente se partieron de los logares oportunos. 7 Assí mesmo Fabio, que tenía sospecha de engaño púnico o carthaginés, no podiendo ser certificado qué cosa fuesse aquello, detovo los suyos en el real. 8 En tanto,  {\persName Hanníbal} pudo passar de la otra parte del monte no lexos de las aguas suessanas por el logar que en este tiempo los moradores de aquella comarca llaman la Torre de los Vaños; y, recogidas todas sus compañas, llegó en salvo al campo Albano y, passado poco espaçio de tiempo, movió el real faza  {\placeName Roma} quasi por la vía derecha. 9 Mas después, buelto el camino, tornose en Apulia; ende se apoderó de un grand pueblo llamado Glereno, logar sin dubda rico y abondoso de todo abasteçimiento de las cosas neçessarias, y quiso allí tener la ynvernada.
\pend
\pstart
36.1 Poco después le seguió el dictador, y no % image:../public/images/1491/171v.png
[171v,b] lexos del real de los  {\name carthagineses} se aposentó en el campo Larinate, donde fue llamado para venir a  {\placeName Roma} a causa de la república, y aviéndose de partir prestamente para allá, dixo al maestro de la cavallería que estoviesse quedo en el real y en manera alguna no veniesse a las manos con los enemigos. 2 Ca, segund que desd’el comienço proposiera en su ánimo, pensava d’esa manera perseverar después de no pelear en batalla {\small\itshape [12batalla: explicitación del lat. \textit{eo}.]}  provocando al enemigo, ni si el enemigo le provocasse venir con él a las manos.
\pend
\pstart
37.1 Mas acaesçió que, después de su partida, Minucio, sin aver memoria de sus mandamientos, salteó çierta gente de los enemigos que  {\persName Hanníbal} avía embiado a recojer trigo y çevada, y andavan esparzidos por los campos y muertos muchos d’ellos, metiolos en el real fuyendo d’él. 2 Luego la fama d’esto se supo en Roma, y las nuevas d’ello fueron cresçiendo como en nombre de victoria, y tanto fueron llenas d’esto las orejas de la muchedumbre, que lo nunca antes de aquel tiempo conteçido, a desora fue egualada la capitanía del maestro de la cavallería con el dictador Fabio. El qual sufrió con grande ánimo la indigna injuria que le fizieran y bolvió al real. 3 Ya avía dos dictadores en un mesmo tiempo, cosa nunca oýda fasta en aquel día y, divididas las compañas entre sí segund la costumbre de los cónsules, eran prefiridos con egual auctoridad a la capitanía del exérçito.
\pend
\pstart
38.1 Mas Marco Minuçio con estas cosas avía conçebido tanta presumpçión, que no toviera osadía  {\persName Hanníbal}, tantas vezes vençedor, de ser tan arrogante como presumió Minutio sin lo consultar con su compañero yr un día a pelear y aduzir sus compañas en tal logar donde, atajadas con çelada que  {\persName Hanníbal} tenía puesta, segund la voluntad de los enemigos, caýan en tierra los romanos y eran asperamente % image:../public/images/1491/172r.png
[172r,a] feridos y no tenían vía alguna por donde escapassen, si Fabio, teniendo más en memoria la salud pública que la injuria reçebida, no socorriera en tiempo. 2 Ca, sobreveniendo a la contienda con compañas rezientes, sin dubda pudo espantar a los  {\name carthagineses} y dio facultad a las legiones romanas que se recogiessen en logar seguro. D’esto conseguió Fabio muy grand fama de virtud y de prudençia, assí çerca de los suyos como çerca de los enemigos. Y cuentan que  {\persName Hanníbal}, buelto a su real, dixo que él vençiera a Minucio, y Fabio le avía vençido. 3 Y el mesmo Minucio, conosçida la prudençia del varón, segund la sentençia de Hesíodo, vio que devía obedeçer al mejor y con todas sus compañas vino al real de Fabio y, despuesto el magistrado con muy honoríficas palabras, saludó a Fabio llamándole padre. Y aquel día fue muy mucho alegre a maravilla a todos los guerreros romanos ende juntos y ambos exérçitos aposentados en sus aparejos para invernar, después de luenga contienda, fueron criados nuevos cónsules Lucio Paulo Emilio y Gayo Terencio Varró, ombre de baxa plebe que el viento popular avía ensalçado a ser cónsul.
\pend
\pstart
39.1 A éstos se dio comissión que ellos, con mayores compañas de las que tenían acostumbrado acabdillar los primeros capitanes, posiessen en obra la empresa romana. Fueron acresçentadas las legiones y otras nuevas añadidas a las viejas. 2 Los cónsules fueron al exérçito y, como eran de diverso ingenio, assí guardavan diversa razón en administrar {\small\itshape [13P. omite el complemento lat. \textit{imperio}.]} . Lucio Paulo, varón prudente, acordándose de la arte y del consejo de Fabio, quería alongar la guerra y detener al enemigo y abstenerse de batalla. 3 Varro, al contrario, estava lleno de furia y de osadía y buscava como peleassen, y dende a % image:../public/images/1491/172r.png
[172r,b] poco espaçio se mostró con grand tribulaçión y perdimiento de la çibdad, quánta differençia oviesse entre la modestia de Emilio y la presumpçión demasiada de Varrón. 4 Ca  {\persName Hanníbal}, temiendo que por falta de mantenimiento de pan se levantasse algund movimiento en el real, ydo de Glereno por llegar a otros logares más calientes de Apulia, se aposentó con todas sus compañas junto a Cannas; y seguiéndole los cónsules romanos, aposentáronse çerca d’él en dos reales, y los unos de los otros estavan tan çercanos, que solo el río \edtext{Aufido}{\Afootnote{; Ausido ()}} {\small\itshape [14Aufido: en la traducción palentina aparece Ausido, quizás por un error de imprenta.]}  los despartia. 5 Aqueste sólo río, segund escriven algunos, divide en dos partes el Apenino y nasçe de la parte que los montes penden al mar Inferior, que es el Tyrrheno, y después da buelta a salir al mar Adriático. 6 Y Lucio Paulo, viendo a  {\persName Hanníbal} que si morasse en tierra ajena no podría luengamente sostener compañas de tan diversas gentes, perseverava en aquel acuerdo de alargar la guerra; y judgava que éste un camino era para que vençiessen los romanos con salud de la república, y que los enemigos pereçiessen; y que si a aquello mesmo podiesse acabar con Terencio Varrón, se pareçía bien que, estando quedas las compañas de los romanos, se podían quebrantar las fuerças y favores de  {\persName Hanníbal}. 7 Mas aquel varón de ánimo no reposado, no sólamente no se movía por la auctoridad de Emilio que le consejava con tanta prudençia, mas aun le increpava, y regañando dizía y divulgava que, saliendo  {\persName Hanníbal} con az ordenada para pelear, Emilio tenía a los romanos ociosos en el real.
\pend
\pstart
40.1 Por ende, llegado el día en que le cabía el soberano cargo de capitanear, porque a días capitaneavan el uno un día y el otro el día seguiente {\small\itshape [15el uno ... día siguiente: amplificación del lat. \textit{uicissim}.]} , en amaneçiendo, después de passado el río \edtext{Aufido}{\Afootnote{; Ausido ()}}, propuso Varrón % image:../public/images/1491/172v.png
[172v,a] señal de pelear sin lo consultar con el compañero, mas aun contra su voluntad, y le seguía de grado porque non podía contradezirle. 2  {\persName Hanníbal}, alegre por tener occasión de pelear, pues que vía que toda dilaçión le era contraria, passó sus compañas a aquella parte del río muy aparejadas y adornadas con todo linaje de armas. 3 Ca los muchos despojos de los enemigos que oviera le dieron materia para adornar su gente. La az de los romanos estava buelta al mediodía y çegávales los ojos el polvo que lançava el viento ábrego, que los de aquella tierra llaman vulturno. 4 Los enemigos miravan contra septentrión y, estavan de tal manera puestos en az, que cada una de las alas o puntas tenían los africanos y lo de medio tenían los galos y los españoles. 5 \edtext{Primero}{\Afootnote{; Premero ()}} fue el concurso de la ligera armadura y luego en pos d’ellos concurrieron los de cavallo y, por quedar poco espaçio entre el río y el peonaje para que peleassen, fue la contienda más feroçe que luenga. 6 Ya arredrada y echada del campo la gente de cavallo de los romanos, restava la batalla entre los peones con tanto ardor de ánimos, que pareçía ningund otro tal tiempo tener para pelea. 7 Mas la grand cobdiçia de vençer, segund que en la primer arremetida tovo el començo alegre, assí después dio triste salida a los romanos, porque los galos y los españoles que arriba he mostrado ser collocados en medio, no podiendo sofrir el ímpeto de los romanos, se recogieron al socorro de los africanos. 8 Y los romanos corriendo apretaron con los enemigos y, mientra que para les más apretar se estriñieron en uno juntamente, dieron occasión a los  {\name \edtext{carthagineses}{\Afootnote{; carhagineses ()}}} que de cada parte los çerrassen con las puntas. 9 Y quinientos cavalleros que poco antes, con simulaçión de ser fuydizos, fueran amigablemente reçebidos por los cónsules y los avían mandado estar entre los postrimeros de % image:../public/images/1491/172v.png
[172v,b] la gente, quando les pareçió tiempo de fazer el negoçio, mostráronse contrarios a las espaldas y de súbito arremetieron contra los enemigos que d’esto estavan descuydados. Entonçes la az de los romanos de toda parte muy turbada ovo de dar la victoria no dubdosa a los  {\name carthagineses}.
\pend
\pstart
41.1 Dizen ser muertos en aquella batalla, segund la auctoridad de  {\persName Livio} , fasta quarenta mil peones y sobre dos mill y sietecientos cavalleros. Polibio pone ser muy mayor el número de los muertos. 2 Pero, dexadas estas cosas en medio, es devido affirmar que los  {\name carthagineses} ninguna mayor tribulaçión, nin tan perdidosa, induxeron a los romanos nin en la primer guerra púnica, nin en esta segunda, que fue la de Cannas. Ca Paulo cónsul fue muerto en aquella batalla, varón digno de memoria y que fasta el postrimer día fuera provechoso a la patria. 3 Fue otrosí muerto Servilio, cónsul del año antepassado, y otros varones consulares, y pretores, y propretores y tribunos militares, y ediliçios y otros muy muchos y muy honestos senadores, y exérçito de oportunos çibdadanos que fasta hartura del muy cruel enemigo cayeron muertos.
\pend
\pstart
42.1 El cónsul Terencio, que fuera auctor de cometer la batalla, quando en todas partes vio ser ventajosos los  {\name carthagineses}, fuyendo buscó salvarse; y Tuditano, tribuno militar, con grand compaña rompió por medio de los enemigos y vino a Canusio, 2 do llegaron fasta diez mill ombres librados de mano de los enemigos como de tempestad, y fue encomendada la suma de la capitanía por consentimiento de todos ellos a Appio Pulcro y a Publio Cornelio Scipión, que después dio conclusión a esta guerra.
\pend
\pstart
43.1 Aqueste fin ovo la batalla de Cannas, y muy prestamente llegada la nueva a Roma, comoquier que hinchió la çibdad de tristeza y de llantos y de lucto, segund % image:../public/images/1491/173r.png
[173r,a] quería la razón. 2 Pero el senado y el pueblo romano no \edtext{solamente}{\Afootnote{; sala mente ()}} tovo cuydado de guardar la çibdad y retener su dignidad en las adversidades, mas aun de reparar el exército, y fizo tomar armas a los más mançebos; y no menospreçió a  {\placeName Sicilia} y a  {\placeName España}, de manera que el que aquesto quisiere considerar, se deve maravillar del grande ánimo y consejo como en aquella tribulación tovo el pueblo romano y del orden que dio en tiempo de tantos trabajos. 3 Ca, dexadas las pérdidas reçibidas en Tiçino y çerca de Trebia y \edtext{de}{\Afootnote{; do ()}} Transimeno, recresçió aquesta postrera llaga con que del todo cayeron los favores del pueblo romano. 4 Las quales cuytas ¿qué otra gente podiera sostoner? Pero sofriolas el pueblo romano y de tal guisa las sostovo, que pareçió no faltar a la grandeza de los ánimos consejo, nin assí mesmo industria para el consejo.
\pend
\pstart
44.1 Y  {\persName Hanníbal} vençedor dio a los vençidos espaçio para respirar mientra que rehazía el exército y gastava tiempo. Ca, si luego en fin de la pelea aduxera el exército vençedor a Roma, sin dubda, o los romanos del todo descayeran, o incurrieran extremo peligro. 2 Escriven que  {\persName Hanníbal} se arrepentió después muchas vezes d’esta tardança y que públicamente se quexava por aver más querido en aquel tiempo consejar al reposo de su gente que creer a Maharbal, cabdillo de los cavalleros, el qual judgava que súbitamente era de yr a Roma, cabeça de la guerra; y dizen que, viendo cómo  {\persName Hanníbal} se detenía, pronunció aquellas palabras tan divulgadas: «Vençer sabes  {\persName Hanníbal}, mas no sabes usar de la victoria». 3 Y es çierto que, segund dixo aquel Néstor çerca de Homero, no son conçebidas todas las cosas juntamente a los ombres. A los unos falta la arte del vençer, a los otros la presteza del concluyr, y assí mesmo a algunos falleçió el estudio del conservar. % image:../public/images/1491/173r.png
[173r,b] 4 Sabemos que Pyrrho, rey de los epirotas, que fizo guerra al pueblo romano, fuesse muy principal capitán de guerra. Pero es puesto en memoria que fue singular varón en aquistar señorío y poco apto en lo conservar. De essa guisa devemos conosçer por las historias antiguas que algunos alcançaron algunas cosas en las guerras dignas de loores y otras les fallecieron.
\pend
\pstart
45.1 Después d’esta batalla avida en Cannas, los atelanos y calatinos y samnites y brucios, y allende d’estos los lucanos y muchos otros pueblos de Ytalia, conmovidos por la fama de tan grand victoria, seguieron la parçialidad de  {\persName Hanníbal} y faltaron a los romanos. 2 Otrosí Capua, que mucho antes lo deseava,  {\persName Hanníbal} de nuevo se coniungió con su compañía dexados los amigos viejos. Y segund la opinión de las gentes, aquesto induxo grand momento a las cosas de los  {\name carthagineses}, 3 porque en aquel tiempo era muy rezia çibdad por muchedumbre de çibdadanos y de labradores y, \edtext{después}{\Afootnote{; despuds ()}} de Roma, era la çibdad que en Ytalia más floreçía entre todas. 4 Y de muchas cosas que d’ella se escriven en pocas palabras comprehenderé que fue colonia de los ethruscos o toscanos, y llamose primero Vulturno, y después Capua, por el nombre de su capitán Capis, o, lo que es más çercano a la verdad, se cree ser llamada Capua por el campo grande y extendido, porque en el derredor d’ella hay campos muy nobles en toda fertilidad de la tierra y en fermosura de las labranças que llaman los griegos {\small\itshape [16P. disimula la ausencia de la palabra en griego.]} . 6 Aquesta comarca de toda parte es çiñida de pueblos ylustres, ca a la costa del mar moran los suesanos y los cumanos y los napolitanos. En lo mediterráneo, a la parte de septentrión, los calentinos y los calenos. A oriente y a parte del mediodía, los daunos y los nolanos. Allende d’esto, el % image:../public/images/1491/173v.png
[173v,a] logar mesmo en que fue fundada Capua es naturalmente guernecido de la una parte el mar y de la otra montes perpetuos que la ciñen. 7 Assí que en aquel tiempo, siendo florecientes los capuanos, quando vieron que en la batalla çerca de Cannas los romanos eran quasi del todo destruydos, de ligero, segund las más vezes se faze, se inclinaron al vençedor, y no sólamente quesieron juntar consigo en compañía al vençedor  {\persName Hanníbal}, mas aun le reçibieron dentro de la çibdad con honor increýble. 8 Porque los çibdadanos speravan que en el fin de la guerra ellos quedarían por más principales y más poderosos y ricos entre los ytalianos. Y en esta manera la sperança y la opinión muchas vezes engaña a los mortales en las cosas humanas.
\pend
\pstart
46.1 Venido  {\persName Hanníbal} a entrar en Capua, saliole a reçebir al camino mui grand muchedumbre, con gana de veer capitán de tan honrado y soberano nombre, cuyas victorias tan grandes y tan bienventuradas batallas por él cometidas se tratavan en la boca y en los ojos de todos. 2 Ya entrado en la çibdad, fue levado a casa de Pacuvio {\small\itshape [17Pacuvio: corrección del lat. \textit{Pacunii}.]} , varón su amigo y muy conosçido y sin contienda prinçipal de los capuanos en poderío y en estimación. 3 Estava aparejada la cena de muy exquisitos manjares. Y de los capuanos ninguno fue allí conbidado a cenar con él, sino Vibellio Taurca, varón muy fuerte, y el fijo del huésped Pacuvio. Y el padre quiso aplacar a  {\persName Hanníbal} que estava enojado con su fijo, porque seguiera a Decio Magio que pugnava porque se guardasse la compañía de los romanos, y tovo aquello Pacuvio por grand negocio. 4 Pero es cosa muy provechosa mirar quántos y quánd diversos peligros algunas vezes acaescen a los varones más principales, sin que los ombres lo piensen. Ca aquel mançebo, aunque % image:../public/images/1491/173v.png
[173v,b] segund la simulaçión pareçía averse reconciliado a  {\persName Hanníbal}, pero attendía occasión de le dañar; 5 y mientra el combite se festejava con razonamiento y alegría de todos, aduxo al padre en la parte más secreta de la casa y exhortávale que quisiesse juntamente consigo redemir con soberano benefiçio la gracia del pueblo romano que avían perdido por grave pecado, y descubriole lo que tenía acordado y determinado en su ánimo, de matar a  {\persName Hanníbal}, enemigo de su patria y de toda Ytalia. 6 Pacuvio, que era varón de tanta auctoridad y padre de aquel fijo, quedó muy espantado de lo que le dizía y, abraçado el mançebo, con mucho trabajo y con muchas lágrimas a penas en fin pudo acabar con ruegos que, quitada la espada, dexasse en su casa estar seguro el huésped. 7 Assí que  {\persName Hanníbal}, venido desde los postrimeros fines de  {\placeName España}, passado por tan luengas tierras con tan gran hueste entre los tiros de los enemigos y librado de las assechanças de los galos y de todo avía escapado, falleçió poco que un mançebo con su mano no le matasse entre los manjares a la mesa.
\pend
\pstart
47.1 Otro día diose a  {\persName Hanníbal} grand concurso del senado capuano, y en aquel ayuntamiento él con muy gratas palabras pudo contentar y hinchir las orejas de los oyentes, prometiendo muchas cosas y muchas amonestando, y los capuanos prestamente las creyeron porque, errados en su opinión, speravan aver el principado de Ytalia. 2 Y d’esta causa tan torpemente se sometieron al carthaginés que, quasi olvidada su libertad, pareçía averle ellos reçebido en la çibdad no por compañero, sino por señor. 3 Y aun allende de otras cosas, demandando  {\persName Hanníbal} que le entregassen a Decio Magio, principal del otro vando contrario, no solamente por servil decreto el senado lo consentió, 4 mas aun padeçió que a vista % image:../public/images/1491/174r.png
[174r,a] del pueblo llevassen atado en cadenas al real aquel varón por se acordar más de la antigua compañía y del dever de la cosa pública como çibdadano affecçionado al bien que de se contentar de las gentes bárbaras.
\pend
\pstart
48.1 Mientra estas cosas se fazían en Capua, Magón, hermano de  {\persName Hanníbal}, fue a  {\placeName Carthago} a denunciar a sus çibdadanos la muy bienventurada victoria y expuso en el senado con muy magnificadas palabras las cosas que  {\persName Hanníbal} avía fecho; 2 y por fazer fe a su razonamiento, mandó esparzir en el portal de la corte los anillos de oro que se avían quitado a los cavalleros romanos. Y el montón d’estos anillos, unos escriven que hinchió la medida de un modio y otros dizen que de tres moyos y medio. 3 Después pidió más gente para suplir el exército y el senado lo concedió con mayor estudio, que después lo embiaron. Ca los  {\name carthagineses}, conmovidos de las cosas presentes, segund qu’el comienço era alegre, assí, proponiendo en sus ánimos qu’el fin de la guerra sería próspero, determinaron favoreçer a  {\persName Hanníbal} y de juntar gente escogida para le embiar, y de perseverar en armas. 4 Y a este consentimiento de todos uno solo,  {\persName Hannón}, repugnava, enemigo perpetuo del vando Barchino, cuyo consejo saludable los  {\name carthagineses}, porque él amonestava la paz segund que otras muchas vezes, mucho más entonçes lo desdeñaron y desecharon.
\pend
\pstart
49.1  {\persName Hanníbal}, después de fecha la compañía con los capuanos, aduxo la hueste sobre la çibdad de Nola, teniendo \edtext{esperança}{\Afootnote{; esperanca ()}} de averla, que gela darían los çibdadanos de su voluntad. Y su opinión non le saliera engañada si el pretor Marcelo, con su súbita venida en la çibdad, no reprimiera la muchedumbre que estava ganosa de la dar. 2 El qual Marcelo, llegados los enemigos al muro, fizo salir por tres puertas gente % image:../public/images/1491/174r.png
[174r,b] de los suyos, y dio en ellos de tal guisa que, matando muchos, los fizo retraer al real. Aqueste es Marcelo, varón muy señalado en la guerra y claro en la gloria militar, que primero con su grandeza de ánimo y de ingenio enseñó a  {\persName Hanníbal} poder ser vençido.
\pend
\pstart
50.1 Después d’esto  {\persName Hanníbal}, determinado de diffirir el negocio de Nola para en otro tiempo, fue sobre Acerras y de ligero tomó la çibdad y la robó. Desde allí con mayor intento fue sobre Casilino, porque conosçía ser logar muy oportuno para dañar a Capua y, no podiendo conmover con ofreçimientos nin con menazas de peligro a los prenestinos que ende estavan en guarnición, dexó pequeña parte de sus compañas en el çerco del logar y él con toda la otra su gente se fue al aposentamiento de la ynvernada. 2 Y escogió la çibdad de Capua por logar más deleytable y más abondante de todo linaje de plazenterías do él estoviesse. Ende el guerrero, acostumbrado a bevir en lo descobierto y de sofrir con ánimo paciente el frío y la fambre y la sed, ofreçiéndosele de continuo diversas maneras de delectaciones, en breve tiempo, de valiente, se fizo covarde y, de fuerte, temeroso y, de solíçito, perezoso y tierno. 3 Ca las blandas delectaçiones corrompen toda la fuerça del ánimo y la inclinación virtuosa, y abaten el ingenio y quitan el consejo, y sobre estas cosas, ¿qué otro daño puede mayor acaesçer al linaje humano? 4 Por ende con razón llama Platón al deleyte çevo de los males. Assí que, en las cosas presentes, más dañaron a los carthagines las delectaciones capuanas que las cumbres de los Alpes y que los armados exércitos de los romanos; 5 y una ynvernada que se passó en floxedad y en terneza, pudo tanto abatir y apagar el vigor de los ánimos, que los guerreros salidos después al campo en el % image:../public/images/1491/174v.png
[174v,a] comienço del verano pareçían aver olvidado toda la virtud de la guerra.
\pend
\pstart
51.1 Passado el ynvierno, bolvió  {\persName Hanníbal} sobre Casilino esperando que los del logar, después de tan luengo çerco, aunque de mal se les fiziesse, vernían en su mano y poderío. Pero ellos, aunque estavan en gran trabajo por tener mengua de todas cosas, con todo, avían propuesto en sus ánimos padeçer y experimentar qualquier extrema necessidad, antes que dexarse venir a confiança de la fe y alvedrío del muy cruel enemigo. 2 Por ende, primero manteniéndose con farro y a la postre con nuezes que los romanos les osavan traer por el río \edtext{Vulturno}{\Afootnote{; Vulturo ()}}, fasta que  {\persName Hanníbal}, apassionado de enojo, tomó el logar por partido que antes nunca avía açeptado.
\pend
\pstart
52.1 Dende en adelante aquesta guerra en que todas cosas avían sucçedido a los  {\name carthagineses} con maravillosa feliçidad quasi alegres y segund las querían, sin les intervenir entre tantas victorias pérdida alguna digna de memoria, conmençó aver en aquel tiempo diversos acaescimientos y caer en suerte de condición variable de los negoçios. 2 Ca la compañía començada con Philippo rey de Macedonia, y la gente embiada a  {\persName Hanníbal} desde  {\placeName Carthago} para supplir su hueste y la toma de Petilia y de Consençia y de otras çibdades en Abruço, avían mucho {\small\itshape [18mucho: adverbio añadido por P.]}  confirmado el ánimo de los  {\name carthagineses}. 3 De la otra parte, los romanos avían recobrado vigor en sus ánimos por las grandes batallas a ellos prósperas en  {\placeName España} y en  {\placeName Sardinia}, rompiendo sus enemigos de tal manera, que tenían muy grand esperanza de fazer bien su negocio. 4 Y tenían buscados en aquel tiempo muy ventajados capitanes, a Fabio Máximo y a Sempronio Graccho y a Marco Marcelo, varón digno de todo loor en la guerra, los quales tan espiertamente administravan la governación romana, que ya sentía  {\persName Hanníbal} como tenía de guerrear con % image:../public/images/1491/174v.png
[174v,b] muy áspero y muy prudente enemigo.
\pend
\pstart
53.1 Lo primero, teniendo él çercada Cumas, con grand matança que en los suyos fizo Sempronio Graccho y los arredró, fue costreñido  {\persName Hanníbal} dexar el çerco. Poco después Marco Marcelo {\small\itshape [19Marco: explicitación del praenomen de Marcelo.]}  en Nola peleó con él en batalla y reçibió  {\persName Hanníbal} {\small\itshape [20Hanníbal: explicitación del término lat. \textit{Poenis}.]}  mucho daño porque de los romanos aun no cayeron mill y de los  {\name carthagineses} más de seys mill fueron muertos y presos. 2 La qual pelea pareçe de quánta importancia fue, pues devemos entenderlo, porque luego  {\persName Hanníbal} se dexó del çerco de Nola y fue en Apulia y aposentó su exército en la ynvernada. 3 D’esto se fizo que los romanos, quasi escapados de una muy grave enfermedad, con soberanas fuerças fuessen contra el enemigo y osassen, ya no solamente defender lo suyo, mas aun tomar lo ajeno. 4 Y mayormente eran determinados de combatir a Capua porque tenían los ánimos resabiosos de la continua y grave injuria quales fezieran los capuanos, que luego, después de la batalla perdidosa de Cannas, se avían passado a los vencedores  {\name carthagineses} en tiempo tan contrario a la república romana, olvidados los muchos benefiçios que antes avían reçebido de los romanos.
\pend
\pstart
54.1 Al contrario, los capuanos, acordándose de su delicto y espantados del nuevo aparejo que los romanos fazían, embiaron a rogar humilmente a  {\persName Hanníbal}, que a la sazón estava en Apulia, que en \edtext{aquel}{\Afootnote{; quel ()}} tiempo tan neçessario socorriesse a la çibdad de Capua su compañera. 2 Y él, sin alguna tardança, vino a más andar desde Apulia a Capua y, puesto su real a Tifata a vista de Capua, difirió para en otro tiempo la guerra, la pestilencia sobrevenida a los capuanos, {\small\itshape [21capuanos: traducción errónea del lat. \textit{Campanis}.]}  que la vedó. {\small\itshape [22la pestilencia ... la vedó: el sentido de la frase no puede ser otro que ‘Aníbal difirió para más adelante una guerra que los campanos consideraban una calamidad inminente’.]}  3 Mas, discurriendo muchas y espessas vezes por el campo Neapolitano, ofreçiósele de nuevo esperança de tomar a Nola por trayción. Era allí en Nola, como en muchas otras çibdades de Italia, diviso el % image:../public/images/1491/175r.png
[175r,a] senado de la plebe. La muchedumbre cobdiciosa de novedades favoreçía a  {\persName Hanníbal}, y los más principales y mejores favorecían al pueblo romano.
\pend
\pstart
55.1 Yendo  {\persName Hanníbal} a tomar a Nola entonçes, como muchas vezes antes, le occurrió en el camino Marcelo con compañas puestas en orden, y en la primera llegada començó la batalla, en la qual, levando mejoría los romanos, con tanta alegría fizo arredrar al enemigo, que assaz pareçió en aquel día ser de todo perdidosa la pelea a los  {\name carthagineses} si los \edtext{cavalleros}{\Afootnote{; covalleros ()}}, segund el conçierto de Marcelo, ydos por diversas vías vinieran en tiempo. 2  {\persName Hanníbal} no sin grand quiebra, reduzidas las compañas a su real, poco después se partió dende y conduxo el exército en el campo Salentino. Ca algunos mançebos tarentinos presos mucho antes en las batallas adversas de los romanos y dexados libres, quesieron acordarse d’ello para refirir las graçias, y dieran esperança a  {\persName Hanníbal} que avría a Tarento por industria d’ellos si se açercasse con su exército. 3 Y él, conmovido con sus ofrecimientos, quiso en todas maneras dar obra a este negoçio por gozar \edtext{de}{\Afootnote{; ge ()}} aquella çibdad marítima, segund que su ánimo mucho antes lo avía cobdiciado. 4 Ca de todas las çibdades marítimas ninguna avía más oportuna que Tarento para ser ayudado con gente y navíos desde Grecia y para acarrear muchas otras cosas complideras al exército que cada día eran menester. 5 Y comoquier que la guarnición ende puesta lo contrariava y fazía que la cosa se alongasse, pero no desistió de su empresa començada  {\persName Hanníbal} primero que Nico y Philomeno, auctores de la trayción, le dieron la çibdad.
\pend
\pstart
56.1 Los romanos retovieron solamente la fortaleza, la qual de tres partes quasi es çercada del mar y la una quarta parte tiene llegada por tierra y guarnecíala % image:../public/images/1491/175r.png
[175r,b] un muro con su cava. 2 De aquella parte, en balde tentó  {\persName Hanníbal} tomarla porque se defendía strenuamente, y él propuso impedir las entradas angostas del puerto Tarentino, pensando que aquesta una sola fuesse la vía para que los romanos, no podiendo aver mantenimientos, se le diessen. 3 Mas parecía obra diffícile {\small\itshape [23P. omite el lat. \textit{futurum}.]}  porque los enemigos posseýan las çerraduras del puerto y las naves estavan encerradas en aquel pequeño espaçio, que avían de ser çercadas las salidas, y convenía sacarlas del puerto subiecto a la fortaleza y traerlas al mar çercano. 4 Y ninguno se fallava entre los Tarentinos que traxesse en medio la razón de explicar este negoçio, y solamente fue uno,  {\persName Hanníbal}, aunque todos lo deseavan, que vio sacar las naves del puerto con pertrechos y que, puestas planchas debaxo, por medio de la çibdad las podían traer al mar. 5 Assí que, puesta diligencia de los ombres, dende a pocos días se levaron las naves al alto mar y las posieron contra las entradas estrechas del puerto.
\pend
\pstart
57.1  {\persName Hanníbal}, recobrada Tarento después de çient años que veniera aquella çibdad en poder de los romanos, dexada la çibdad a los tarentinos y la fortaleza çercada, bolvió a Samnio, porque los cónsules romanos avían tomado y robado los cogedores del pan del campo capuano y avían aduzido las legiones sobre Capua y posieran ya todos los aparejos de guerra para tomar la çibdad. 2 Assí que  {\persName Hanníbal}, con grande estudio que tenía de emprender el cargo de amparar de Capua, fue con todas sus compañas contra los cónsules romanos {\small\itshape [24los cónsules romanos: traducción del lat. \textit{hostem}.]} , y los romanos, que no recusavan la batalla, de ambas partes salieron ganosos de pelear. 3 De manera que pareçía aver de fazerse allí una señalada cosa de armas si no despartiera el tal intento Sempronio Graccho, que en el mesmo tiempo aduzía consigo en Campaña la gente que reçibiera de Gneo Cornelio que se perdiera % image:../public/images/1491/175v.png
[175v,a] en Lucania. 4 Vistas desde lexos estas compañas ante conosciessen quién fuesse el que venía, aterrecidos los romanos y no menos los  {\name carthagineses}, bolvieron sus gentes al real. Después d’esto los cónsules, por desviar de Capua a  {\persName Hanníbal}, partiéronse a diversas comarcas, el uno se fue en Lucania y el otro a Cumas.
\pend
\pstart
58.1  {\persName Hanníbal}, ydo en Lucania, ovo occasión de pelear con Marco Centenio, el qual, señalado en temeridad y en loca osadía, puso en poder del muy cauteloso enemigo el exército que el senado poco cuerdamente d’él avía confiado. Cometida la batalla, Centenio, peleando muy atrevidamente, fue allí muerto y de los otros escaparon pocos. 2 Llegose a este grand daño, que  {\persName Hanníbal} poco después, yendo en Apulia, puso assechanças a otro exército de los romanos que capitaneava el pretor Fulvio, y desbaratolo y del todo lo echó a perder. Ca de veynte mill ombres escassamente guarrecieron dos mill y todos los otros fueron muertos a voluntad del muy cruel enemigo {\small\itshape [25enemigo: traducción del lat. \textit{ducis}.]} .
\pend
\pstart
59.1 Los cónsules en tanto, tomada occasión de la partida de  {\persName Hanníbal}, bolvieron sobre Capua con todas las compañas y posieron çerco a la çibdad. Y quando lo supo  {\persName Hanníbal}, con el exército ahorrado vino a Capua y, en llegando, arremetió al real de los romanos teniendo ante avisados a los capuanos que en el mesmo tiempo saliessen a pelear contra ellos. 2 Los capitanes romanos, al primer rebate, partieron entre sí la gente y fueron contra los unos y los otros. Ovo poco de fazer en ençerrar los capuanos dentro de la çibdad; la más dura contienda fue contra  {\persName Hanníbal}, el qual, segund que otras muchas vezes, no menos aquel día se ovo en la batalla como muy áspero capitán y aún tentó si podiesse opprimir a los enemigos por algund engaño. 3 Ca los suyos porfiando de % image:../public/images/1491/175v.png
[175v,b] entrar en el real de los romanos, embió uno enseñado en la lengua latina que a alta boz denunciasse a los romanos en nombre de los cónsules, mandando que, pues el real era ya çerca de tomado, fuyessen a los montes ende çercanos. 4 La boz assí súbita ligeramente podiera mover a los oyentes, si los romanos, ya acostumbrados a los engaños púnicos, no conosçieran ser engaño. Assí que, prestamente refirmando, se costriñieron al enemigo que se retraxesse y se recogiesse en su real.
\pend
\pstart
60.1  {\persName Hanníbal}, ya tentadas todas las maneras, quan vio que ninguna facultad le quedava para fazer que los capitanes romanos se arredrassen del çerco de Capua, ansioso por el peligro de los compañeros, quiso confuyr al consejo mucho antes pensado y como reservado para la postre. 2 Aparejadas todas las cosas necessarias, movió su real y con el mayor silencio que pudo passó el río Vulturno y, por el campo Sidicino y Allifano y Cassinate, con señas tendidas, seguió el camino de Roma, pensando que por ninguna otra vía se podría alçar tan porfiado çerco.
\pend
\pstart
61.1 Aquesto, después que fue en Roma notificado por mensajeros çiertos, a desora les salteó tan grande espanto, que jamás otra vez o muy pocas ovo dentro de la çibdad tan terrible turbación. 2 Ca miravan cómo el muy áspero enemigo que tantas vezes avían experimentado en los grandes males de la república, venía sobre la patria con señas enemigables, y qu’el que ellos no podían sofrir aun absente, venía presente a menazar servidumbre al senado y al pueblo romano. 3 Por ende, en cosa tan turbada, plógoles que Fulvio Flacco, uno de los capitanes romanos, se \edtext{llamasse}{\Afootnote{; llemasse ()}} y veniesse de Capua, y los nuevos cónsules Sulpicio Galba y Cornelio Centimalo se aposentassen fuera de la çibdad; 4 Y Gayo Calphurnio pretor toviesse la guarda del % image:../public/images/1491/176r.png
[176r,a] Capitolio con muy firme guarnición de gente, y que los çibdadanos que oviessen fasta entonçes tenido soberanas dignidades, con su auctoridad y mandamiento apaziguassen y sosegassen los ruydos o temores escandalosos que a desora recresciessen {\small\itshape [26apaziguassen ... recresciessen: la frase latina \textit{repentinos... saedarent} ha supuesto una dificultad para el traductor, pues ha duplicado tanto el verbo lat. \textit{saedarent}, como \textit{tumultus} (por 'los ruydos o los temores escandalosos'), y ha resuelto, en fin, el \textit{repentinos} con 'que a desora recresciessen'.]} .
\pend
\pstart
62.1  {\persName Hanníbal}, continuado su camino, no reçibió reposo primero que fue llegado al río Amniene tres millas de  {\placeName Roma} y puso ende su real. Dende a poco, con dos mill de cavallo se fue acercando a la çibdad y, passando desde çerca de la puerta Collina fasta el templo de Hércules con arremetida a cavallo, no solamente pudo veer el sito y muros de tan extendida çibdad, mas aun de lo contemplar con sosiego. 2 Y viendo aquesto, Fulvio Flacco no pudo comportar tan indigna demasía. Assí que muy presto fizo salir cavalleros romanos contra enemigo, los quales, segund les era mandado, con grande ímpeto començaron la pelea y de ligero podieron arredrar a los  {\name carthagineses} de aquel logar.
\pend
\pstart
63.1  {\persName Hanníbal}, el día seguiente fizo salir su exército del real y puso en orden la az, no queriendo fazer alguna tardança en travar batalla, si podiesse atraer a los \edtext{enemigos}{\Afootnote{; euemigos ()}} a que peleassen; lo mesmo proposieron los romanos fazer y cometer la batalla. 2 Y de cada parte salieron al campo las compañas puestas en armas con tan grande alegría de ánimos, que mostravan no recusar peligro alguno por conseguir victoria en aquel día. 3 Porque los  {\name \edtext{carthagineses}{\Afootnote{; carchagineses ()}}} pensavan consistir en aquella batalla quasi postrimera el imperio de toda la tierra del mundo, y los romanos avían de pelear por la patria y por la libertad y por todos los bienes y por veer, dende a poco, si todo aquello avía de ser {\small\itshape [27P. omite el lat. \textit{futura}.]}  en su poderío o de los enemigos.
\pend
\pstart
64.1 Mas acaesçió cosa mui digna de memoria, ca atendiendo ya en armas la señal de la contienda, a desora recresçió lluvia tan rebatada % image:../public/images/1491/176r.png
[176r,b] y tan rezia con terrible tempestad y con tanta fuerça y aspereza, que cada una de las partes fue costriñida reduzir su gente al real. 2 Luego otro día siguiente en que pareçía averse de pelear, pues aquel poco tiempo se avía dilatado, salidos de nuevo y puestos en az los unos y los otros, se levantó otra semejante tempestad de la del día passado, la qual no menos que la de antes affligió assí a los romanos como a los  {\name carthagineses} y los compelió a olvidar la instante contienda para que solamente pensassen de se salvar y fuyr. 3 Viendo  {\persName Hanníbal} estas cosas, dizen que se bolvió a los suyos dando bozes que la una vez no tovo gana de aver a  {\placeName Roma} y aquesta otra vez no se le otorgava facultad para la tomar. 4 Otrosí púsole turbación que, estando él con tantas compañas de cavallo y de pie en tanta çercanía, apretando la çibdad, embiavan los romanos gente en  {\placeName España} para suplir el exército que allá tenían, y que supo como el campo en que él se aposentó se compró por mayor preçio de lo que la razón de su valer requería. Y d’esto indignado, el cambiador fizo mandar por pregón que se vendiessen las boticas de los çibdadanos romanos que allí tenían. 5 Assí que después d’esto  {\persName Hanníbal}, o considerando en su ánimo quand diffícile era tomar a  {\placeName Roma} por fuerça y combate, o temiendo la falta de los mantemientos (ca él solamente veniera proveýdo por diez días), determinó mover el real y, partido de allí, fuese aposentar al luco o selva sagrada de Feronia y fizo robar el templo que dizen ser entonçes ende muy rico, y dende a poco fue camino a Abruço y de Lucania.
\pend
\pstart
65.1 Aquesto quando lo sopieron los capuanos, desesperados de bien negociar las cosas de su estado, dieron la çibdad a los romanos. D’esta manera, avida Capua y reduzida % image:../public/images/1491/176v.png
[176v,a] en poder de los romanos, induxo gran momento y gran deseo en todos los pueblos de Ytalia de innovar las cosas. 2 Otrosí movió los ánimos de las gentes el mesmo enemigo  {\persName Hanníbal}, que començó mandar que muchos logares que él no entendía poder sostener y ampararse robassen y derribassen por mal consejo suyo. Porque antes, aviendo vencido muchas vezes, dexava libres a muchos captivos sin rescate y con aquella liberalidad atraxera las voluntades de muchos pueblos. 3 Y assí entonces una su inhumana crueldad fue causa que muchas çibdades enojadas del señorío púnico bolviessen en la de los romanos. En el número de las quales escriven que fue Salapia, que por mano de Blaçio, principal del un vando, se dio al cónsul Marcelo y en ella fue destroýda y quasi rematada con matança una ala de muy escogidos cavalleros que ende estavan en guarniçión. 4 Aquesta es la çibdad en que algunos escriptores dizen que  {\persName Hanníbal} tenía una amiga que mucho le captivava y, por ello, accusan su destemprada cobdiçia. Hay otros que ensalçando la modestia d’este capitán dizen que nin al comienço mientra fizo guerra en Ytalia, nin después que bolvió en  {\placeName África}, quiso jamás cenar recostado, nin bevió más de un quartillo de vino. 5 Otros algunos historiadores se fallan que attribuyen a  {\persName Hanníbal} la crueldad y el quebrantamiento de la fe y otros vicios de aquella gente, mas ninguna mençión fazen de castidad o luxuria del tal varón. Solamente dizen que tenía muger legítima natural de Castulón, que es noble logar en  {\placeName España} y, por la singular fe de los que en aquella çibdad moravan, tenían con ellos los  {\name carthagineses} principal amor y confianza.
\pend
\pstart
66.1 Y segund avemos de suso mostrado,  {\persName Hanníbal}, des que ovo perdido a Salapia, poco % image:../public/images/1491/176v.png
[176v,b] después ovo de fallar occasión cómo los romanos reçebiessen otro muy maior daño. 2 Ca por el mesmo tiempo Fulvio procónsul estava aposentado çerca de Herdonea, atendiendo que sin alguna contienda podría aver la çibdad, mas sentiendo que podría estar syn temor, pues que ningún enemigo le era çercano (porque  {\persName Hanníbal} era ydo al campo de Abruço), era negligente en fazer velar y en poner guardas y escuchas y en \edtext{toda}{\Afootnote{; todo ()}} otra administraçión de la guerra y acostumbrava floxedad indigna de capitán romano. 3 Y  {\persName Hanníbal}, después que por occultas espías conosçió esta negligencia, pensó que no fuesse de dexar tal occasión de fazer bien el negoçio. Assí que, ydo en Apulia con compañas ahorradas, vino a Herdonea con tanta presteza, que le faltó poco de poder tomar a Fulvio dentro de su aposentamiento syn que en cosa d’esto pensasse. 4 Con todo, los romanos con grand fiuza de ánimos reçibieron la primera acometida y pelearon luengo tiempo de lo que la razón quería; al fin, segund que dos años antes çerca de aquel logar otro Fulvio fue desbaratado, assí aun por desdicha d’este Fulvio procónsul fueron rompidas y vencidas las legiones romanas y muerto el mesmo capitán Fulvio con gran parte de las compañas.
\pend
\pstart
67.1 En el mesmo tiempo \edtext{el}{\Afootnote{; e ()}} cónsul Marcelo estava en Samnio y, quando supo averse reçebido aquella grand pérdida por imprudencia del capitán, aunque pareçía poder dar tardío el soccorro a las cosas ya perdidas, 2 pero con deseo que si se acercasse la ayuda, sería remendar algo de la quiebra, aduxo el exército en el campo Lucano, do era  {\persName Hanníbal} ydo después de la victoria; y, puesto su real a vista de los enemigos, dende a poco se puso en orden de batalla, nin pensaron los  {\name carthagineses} que deviessen fuyr la pelea. 3 Assí que la batalla se cometió % image:../public/images/1491/177r.png
[177r,a] luego con tan grand obstinatión de ánimos, que quasi con eguales fuerças pelearon fasta el sol puesto y la noche pudo despartir la dubdosa victoria. 4 Otro día los romanos, salidos otra vez ordenados en az, fizieron manifestar al enemigo la confessión del temor. Ca  {\persName Hanníbal} tovo a los suyos dentro del fossado de su real y en la noche seguiente, caminando en silencio con sus compañas, vino en Apulia.
\pend
\pstart
68.1 Otrosí Marcelo, proseguiendo los passos de  {\persName Hanníbal}, buscava occasión cómo peleasse con él en alguna memorable batalla para aclarar la summa de la guerra que ya se tenía por dicho que él mayormente, entre los capitanes romanos en consejo y diligencia y disciplina y en todo otro officio militar, se podía comparar a  {\persName Hanníbal}. 2 Mas fue impedimento el tiempo de la ynvernada que ya instava, para que no oviesse logar de contender con el enemigo en as ordenada y justa, mas, cometidas algunas escaramuças, ovo de yrse a aposentar en las estançias del invierno porque no pareciesse que en balde fatigava su gente de guerra.
\pend
\pstart
69.1 En el commienço del verano, en parte incitado por letras de Fabio, que era uno de los cónsules de aquel año, y en parte de su natural desparteza {\small\itshape [28su natural desparteza: traducción por amplificación del lat. \textit{natura sua}.]} , Marcelo, más presto de lo que todos pensavan, salió del aposentamiento de la ynvernada y adduxo su exército contra  {\persName Hanníbal}, que en aquel tiempo estava aposentado junto a Canusio. 2 Y acaesció que, por la çercanía de los reales y por la gana de pelear, en pocos días pelearon tres vezes con señas desplegadas en batalla. 3 En el primer día, \edtext{pelearon}{\Afootnote{; peleeron ()}} quasi a la eguala cada una parte con esperança de vencer, fasta que los despartió la noche sin que inclinasse la victoria más a los unos que a los otros, y, assí como de acuerdo, cada una de las partes se reduxo a su real. 4 El segundo % image:../public/images/1491/177r.png
[177r,b] día levó ventaja  {\persName Hanníbal} y, muertos çerca de dos mill y sietecientos de los enemigos, fizo bolver como fuyendo a las otras compañas de los romanos. 5 El terçero día los romanos, cobdiciosos de se limpiar de la infamia qu’el día de antes incurrieron por la quiebra recebida, primero demandaron que querían pelear. Y Marcelo los puso en az, de la qual ferocidad de Marcelo fue  {\persName Hanníbal} mucho maravillado, y escriven que dixo que tenía contienda con varón que ni vencedor nin vencido podía reposar. 6 Luego fue entre ellos más áspera y más dura batalla que alguna de las batallas. Ca los romanos se trabajavan de emendar el daño reziente que recibieran. Y los  {\name carthagineses} estavan indignados que entonces los vencidos se adelantassen a provocar a los vencedores a pelear. 7 Al fin, los romanos, increpados y amonestados por Marcelo que diessen obra como antes llegasse a  {\placeName Roma} la nueva de su victoria que de su primero daño, apretaron más agramente, no fizieron primero fin de pelear, fasta que, trasdoblada la pérdida de sus enemigos, los fizieron fuyr.
\pend
\pstart
70.1 En el mesmo tiempo, quasi por aquella manera que se perdió Tarento, la recobró Fabio Máximo. Y denunciada la nueva a  {\persName Hanníbal} como Tarento era tomada por el cónsul Fabio, escriven que dixo: «Los romanos tanbién tienen su  {\persName Hanníbal}».
\pend
\pstart
71.1 En el año seguiente, Marcelo y Crispino, criados cónsules y con todas las fuerças empleados en fazer la guerra, conduxeron dos exércitos consulares contra los enemigos. Y  {\persName Hanníbal}, no confiando que los podiesse resistir a la eguala en el campo, segund que antes muchas vezes, assí mesmo en aquel tiempo pensó poner todo su ingenio en tomar su enemigo por assechanças engañosas, pues no le podía vençer en pelea del campo. 2 Mientra él tenía este pensamiento, % image:../public/images/1491/177v.png
[177v,a] ofreciósele mucho mayor occasión de fazer bien su negoçio de lo que él tenía pensado en su osadía. Estava un otero silvestre entre medias de los dos reales de los  {\name carthagineses} y romanos, y debaxo de aquel otero estavan escondidas por mandado de  {\persName Hanníbal} algunas manadas de Númidas, puestas en logares oportunos para que si algunos de los enemigos se esparziessen por los campos los atajassen. 3 De la otra parte, los cónsules, por commún boz y acuerdo de todos, se movieron a mirar aquel otero qué tal era, porque, si menester fuesse, pareçía que lo deviessen occupar, porque si ellos lo dexassen, non lo tomassen los enemigos, y después estoviessen allí sobre sus çervizes. 4 Pero ante que determinassen mover el exército por atalayar la natura del logar, salieron ambos del real e yendo allá con pocos de cavallo más desacordadamente de lo que pertenecía a \edtext{tan}{\Afootnote{; tam ()}} principales varones, cayeron en las assechanças que les estavan aparejadas. 5 Assí que, súbitamente atajados, no podieron salir rompiendo a la fruente por medio de los enemigos, y por las espaldas le firían. De manera que, más por necessidad que por consejo, començaron batallar. En la qual pelea, fue muerto Marcelo lidiando muy valientemente y fue ferido Crispino, que a penas se pudo escapar de mano de los enemigos, segund el poco espaçio que le davan.
\pend
\pstart
72.1 Y  {\persName Hanníbal}, después que supo como era muerto Marcelo, el qual principalmente entre los capitanes romanos reprimiera el curso de sus victorias, súbitamente mudó su real en el collado do avía sido la pelea; 2 y allí fallado el cuerpo de Marçelo, fízole sepelir con muy grand honor. Donde se \edtext{puede}{\Afootnote{; puedo ()}} comprehender que tan grand valor se estima çerca de todo linaje de ombre la grandeza de ánimo y la exçellencia de la virtud, pues qu’el muy cruel enemigo, % image:../public/images/1491/177v.png
[177v,b] viendo muerto a tan prinçipal capitán, no padeçió que su cuerpo careçiesse de sepultura honrosa.
\pend
\pstart
73.1 En tanto, los romanos, quando vieron muerto el un cónsul y el otro ferido, retraxéronse prestamente en los montes más çercanos y aposentáronse en logar seguro. Y ya el mesmo Crispino avía embiado amonestar a las çibdades vezinas que, por ser muerto su compañero y tener el enemigo su anillo, se guardassen de letras compuestas en nombre de Marçelo. 2 Ya era llegado a Salapia un mensajero de Crispino, quando llegavan ende letras embiadas por  {\persName Hanníbal} que, en nombre de Marçelo, denunciavan como él vernía ende la noche seguiente. Y los salapitanos, que conosçieron el engaño de  {\persName Hanníbal}, tornáronle embiar el mensajero y con muy atento cuydado attendieron la venida de los enemigos. 3 Llegó  {\persName Hanníbal} a Salapia en la quarta vigilia de la noche, y de industria yvan en la delantera los fuydizos de los romanos, porque fablando la lengua latina fiziessen fe que Marcelo venía allí. Y los de logar, quando ovieron reçebido dentro fasta seyscientos ombres, cerraron las puertas y fizieron desviar con tiros las otras compañas ende llegados y mataron todos los que primero avían reçibido dentro. 4 De manera que  {\persName Hanníbal} quedó enojado del baldío camino que avían començado y movió su real y fuese en Abruço, porque assí por tierra como por mar socorriesse a los locrenses que estavan çercados.
\pend
\pstart
74.1 Después d’esto, con muy grand diligencia y cuydado de los padres y de la plebe, fueron fechos dos nuevos cónsules muy enseñados capitanes en la guerra, Marco  {\persName Livio} y Claudio Nero. 2 Los quales, partidas entre sí las compañas, fueron a sus provincias: Claudio en los salentinos y  {\persName Livio} en Galia, contra  {\persName Hasdrúbal} Barchino, el qual, ya passados los Alpes, dava priessa % image:../public/images/1491/178r.png
[178r,a] de se juntar con su hermano con muy crescidas compañas de peones y cavalleros.
\pend
\pstart
75.1 Acaesçió en el mesmo tiempo estoviesse  {\persName Hanníbal} muy affligido de los daños que le fazía el cónsul Claudio, el qual primero en Lucania con assechanças segund el arte púnica pudo desbaratar al enemigo. 2 Después en Apulia, çerca de Canusio, peleó con  {\persName Hanníbal} en escaramuça con tanto ímpeto que pudo matar grand número de los enemigos; y, con esta quiebra, fue tan apassionado  {\persName Hanníbal}, que luego se fue a Metaponto por reparar su exército. 3 Y deteniéndose allí pocos días, recibió la gente que  {\persName Hannón} tenía, y juntolas con las suyas y bolvió otra vez a Venusia. 4 Y Claudio tenía no lexos de Venusia su real y, en tanto, fueron tomadas en el camino por los de Claudio letras por las quales supo Claudio {\small\itshape [29Claudio: duplicación del sujeto.]}  que  {\persName Hasdrúbal} se acercava. 5 Y en su ánimo días y noches pensava en qué manera podría fazer que no se juntassen en uno las compañas de tan grandes capitanes. Assí que, considerándolo todo por razón y echada cuenta, ovo de tomar consejo peligroso, segund que pareçía que sería, pero por ventura necessario en aquel tiempo. 6 Ca dexado el real en guarda del legado suyo o comissario, él, con parte del exército, a grand caminar por vía del Piceno llegó a Sena en seys días. Y allí ambos los cónsules juntaron sus compañas çerca del río Methauro y pelearon con  {\persName Hasdrúbal} en batalla muy bienaventurada, en que dizen que fueron muertos aquel día fasta cinquenta y seys mill de los enemigos, de manera que cuenta ser çerca egual aquella pérdida de los  {\name carthagineses} a la que los romanos ovieron cabe Cannas.
\pend
\pstart
76.1 Y Claudio Nero, después d’esta tan memorable victoria, no con menor priessa que veniera bolvió a Venusia, y mandó soltar a unos que tenía captivos porque posiessen la cabeça \edtext{de}{\Afootnote{; des ()}}  {\persName Hasdrúbal} % image:../public/images/1491/178r.png
[178r,b] junto a la estancia de los enemigos, y que levassen a  {\persName Hanníbal} nueva de tan grand cuyta y tribulación. 2 Ca sábese no aver primero sabido  {\persName Hanníbal} lo que primero fuera aparejado por consejos occultos y las cosas que en los días ante passados se fizieran. En esto es de aver por maravilla que la grand solicitud de capitán tan cauteloso fuesse engañada por Claudio, siendo tan pequeño intervallo entre ambos reales, y que primero supiesse ser muerto su hermano con todas sus compañas, que sentiesse ser partido dende y tornado al real el cónsul romano. 3  {\persName Hanníbal}, lastimado d’esta llaga, non solamente pública mas suya propria, dixo que él ya conosçía el retorno desaventurado de las cosas de los  {\name carthagineses}, y dende a poco se partió de allí y se fue al campo de Abruço. 4 Ca bien conosçió el muy enseñado capitán quánto favor se avía de recrescer y quanto importava a la empresa romana para en toda la guerra aquel tan terrible daño de los  {\name carthagineses} reçibido çerca de Methauro.
\pend
\pstart
77.1 Pero él, recogidas todas las fuerças que le quedavan en Ytalia, después de tantas batallas y çibdades tomadas por fuerça, con ánimo no vencido sostovo por aquel tiempo la guerra y lo que a qualquier ombre deve paracer cosa maravillosa, o por auctoridad o por prudencia, tovo tan concorde el exército aduzido de españoles, africanos y galos y mesclado de otras gentes, que nunca una pequeñita discordia militar ovo en sus reales; 2 y nunca los \edtext{romanos}{\Afootnote{; romnos ()}}, que ya avían recobrado a  {\placeName Sicilia} y a  {\placeName Sardinia} y a las  {\placeName Españas}, podieron rematar este enemigo en Ytalia o sacarle d’ella antes que embiassen en  {\placeName África} a Publio Cornelio Scipión, que, faziendo guerra a los  {\name carthagineses} desde çerca, puso en tan grand peligro a  {\placeName Carthago}, que luego por necessidad los costriñió % image:../public/images/1491/178v.png
[178v,a] revocar de Ytalia a  {\persName Hanníbal}.
\pend
\pstart
78.1 Estava en aquel tiempo, segund antes es dicho,  {\persName Hanníbal} en el campo de Abruço, y fazía la guerra más por incursiones que por peleas ordenadas, sino algunas vezes que travó escaramuça con el cónsul Sempronio, y dende a no mucho espaçio peleó con él en batalla. En la primera contienda levó ventaja  {\persName Hanníbal}, en la segunda el cónsul romano. 2 Adelante después d’esto no fallo escripto por los auctores griegos nin latinos que  {\persName Hanníbal} fiziesse en Ytalia cosa digna de memoria. Llamado que veniesse en  {\placeName África} por mandado de los  {\name carthagineses}, passados diez y seys años d’esta guerra púnica, ovo dexar a Ytalia, querellándose mucho primero del \edtext{senado}{\Afootnote{; senada ()}} de los  {\name carthagineses} y mucho tanbién de sí mesmo. 3 Del senado, porque en tan luengo tiempo como él se detovo en tierra de los enemigos, le avían ayudado con poca gente para suplir su exército y no con dinero o con otras cosas que para la guerra son necessarias y se requerían. 4 Y quexávase de sí mesmo, porque, rompidas tantas vezes las legiones romanas y muertos tantos enemigos, siempre alongando después de la victoria, diera espaçio al enemigo que respirasse. 5 Es puesto en memoria que antes de su subida en el navío para se partir con la flota, fizo un arco çerca del templo di Junón Lacinia, esculpido con letras púnicas y griegas y latinas, en que se contenían sumariamente las cosas que fiziera magníficas.
\pend
\pstart
79.1 Partido  {\persName Hanníbal} de Ytalia y avida assaz próspera navegatión, en pocos días arribó a Lepti, y, puestas todas las compañas en tierra, fue primero a Adrumento y dende a Zama. Y sabido el estado de las cosas carthaginesas, determinó ser más provechoso començar acuerdo de dar fin a la guerra; 2 y plogole % image:../public/images/1491/178v.png
[178v,b] embiar mensajería a Scipión a le requerir que eligiesse algún logar en medio, para que entre ellos fablassen, porque él quería comunicar con él cosas muy principales. 3 No es del todo sabido si  {\persName Hanníbal} fizo aquesto por mandado del senado o por su particular acuerdo. Pareció a Scipión que no tenía causa para refutar la fabla. 4 Por ende, el día que entre sí acordaron, venieron a fablar los tan honrados capitanes de tan poderosas gentes en campo descobierto cada uno d’ellos con un solo intérprete, y por diversos pareçeres razonaron de la paz y de la guerra. 5 Ca  {\persName Hanníbal} mayormente avía inclinado el ánimo a la paz, porque mirava cómo los negocios de los  {\name carthagineses} de cada día empeoravan, aviendo perdido a  {\placeName Sicilia}, y a  {\placeName Sardinia} y las  {\placeName Españas}, y transpassada la guerra desde Ytalia en  {\placeName África}, y aviendo los romanos prendido a Siphace, rey muy poderoso. 6 Assí que la esperança de la salud de los  {\name carthagineses} toda estava puesta en aquellas compañas que  {\persName Hanníbal} consigo tenía y avía traydo de Ytalia como reliquias de guerra tan prolongada, y si algo de más duro desastre les conteciesse, apenas remaneçía a los  {\name carthagineses} tanta fuerça de lo de fuera o de lo de la çibdad mesma que podiesse bastar para defender los muros. 7 Por ende, usando  {\persName Hanníbal} de luengo razonamiento, mucho y con grand vehemencia contendió por atraer en aquella sentencia a Scipión, que quesiesse más la paz que la guerra. Mas Scipión, que tenía grande esperança de rematar la guerra, mostró desviarse de los consejos de la paz. 8 Assí que, después de luenga fabla de la una parte a la otra, partiéronse dende sin concluyr cosa alguna. Y no passó grand espaçio que se cometió aquella batalla memorable que ambos los capitanes entre sí ovieron çerca de Zama, que fue muy mucho bienaventurada a % image:../public/images/1491/179r.png
[179r,a] los romanos.
\pend
\pstart
80.1 En la primer \edtext{arremetida}{\Afootnote{; arrematida ()}} los elephantes de los  {\name carthagineses}, bueltos contra los suyos, turbaron toda la cavallería de  {\persName Hanníbal}, y Lelio y Maxinissa desde ambas las alas o puntas, acrescentándoles el miedo, ningund espaçio dieron a los cavalleros para se recojer del desbarato. 2 Pero los peones de cada una de las azes pelearon luenga y agramente unos contra otros. Ca los  {\name carthagineses}, confiando de las victorias passadas, pensavan ser colocada la salud de toda  {\placeName África} en sus diestras. Y los romanos, siendo eguales en el ánimo, eran soberanos en la esperança. 3 Mas a los romanos no fue pequeño momento para que venciessen que Lelio y Maxinissa, después de desbaratados los cavalleros, bolvieron a la batalla oportunamente y posieron muy grande espanto a los enemigos. 4 Ca, llegando ellos prestamente, se fizo que los  {\name carthagineses} perdiessen el primer ardor de la pelea y no buscassen otro remedio a su salud salvo fuyr. 5 Aquel día mataron los romanos de los enemigos más de veynte mill y çerca de otros tantos se cuenta ser presos. Y el mesmo capitán  {\persName Hanníbal} deteniéndose fasta el fin de la batalla, ya a la postre d’entre la matança de los suyos ovo de fuyr con pocos.
\pend
\pstart
81.1 Desdende, llamado qu’él legasse a  {\placeName Carthago} y socorriesse a la república descayda, mostró al senado que ya no quedava esperança que deviessen reponder en las armas en lo de adelante. Y amonestoles que, todas otras cosas pospuestas, embiassen embaxadores al capitán romano, los quales podiessen con qualquier razón concluyr paz. Embiaron \edtext{diez}{\Afootnote{; dizen ()}} embaxadores que bolvieron a  {\placeName Carthago} con las condiciones de la paz. 2 Dizen que un Gisgón, contradiziendo a la paz, dixo su parecer de renovar la guerra contra los romanos y que  {\persName Hanníbal}, indignado porque en tal tiempo los ombres ignorantes % image:../public/images/1491/179r.png
[179r,b] dizían tales demasías {\small\itshape [30dizían tales demasías: traducción por amplificación del lat. \textit{talia iactari}.]} , derribó del logar soberano al que estava razonando; 3 y, porqu’el tal negocio pareció a la muchedumbre ser osado e indigno de çibdad libre, subió él en el púlpito y dixo que no convenía indignarse, pues él, no aviendo estado en  {\placeName Carthago} desde la primer puericia, y gastada toda su edad en armas y en la guerra, no tenía sabidas las costumbres de la çibdad.
\pend
\pstart
82.1 Después d’esto, con tanta prudencia dixeron sobre las condiciones de la paz, que luego los  {\name carthagineses}, conmovidos por auctoridad de tan grand varón, determinaron acceptar las leyes qu’el vencedor y la necessidad les dava. 2 Las condiciones fueron muy duras quales acostumbraron los vencedores cargar sobre los vencidos en las cosas postrimeras. Pero allende de todo lo ál, eran tenidos los  {\name carthagineses} fasta çierto tiempo pagar tributo cada un año a los romanos. 3 Y llegado el día que la primera paga se devía fazer a los romanos y todos gemiessen, oýdo el nombre del tributo, dizen que, conmovido  {\persName Hanníbal} por la inútile lamentación de los  {\name carthagineses}, començó reýrse, 4 y increpole  {\persName Hasdrúbal} Heduo porque tan de manifiesto se mostrava alegre de la común tristeza de toda la çibdad. Y respondió que no era risa aquella de ombre que se alegrava, mas de quien escarnecía las vanas lágrimas, de aquellos que en aquel tiempo en el más ligero mal tocante al dinero de cada un ombre privado, manavan más que antes quando los romanos quitavan a los  {\name carthagineses} la flota y las armas y los despojos de tan extendidos vencimientos y ponían leyes a los vencidos.
\pend
\pstart
83.1 No tengo en olvido aver auctores que digan como  {\persName Hanníbal}, temiendo que a petición de Scipión gele entregassen, se partió fuyéndose en Asia luego que la contienda de la batalla se fizo mal. Mas si aquesto se fizo luego súbito o algund tiempo después de la pelea de Zama, pensamos % image:../public/images/1491/179v.png
[179v,a] no ser de mucha importancia pues assaz consta que, los negocios venidos en desperación, se fue en Asia y, penado por destierro, vino a Anthíoco. 2 Otrosí es fama no dubdosa qu’el rey le reçibió con hospitalidad tan honrosa, que luego juntamente le puso en los consejos públicos y privados. 3 Era el nombre de  {\persName Hanníbal} de grand gloria çerca de todos y allende d’esto se llegava la commún ira y enemistad contra los romanos y muy grand pungimiento o aguijón para incitar la guerra. 4 Por ende parecía que llegasse a tiempo oportuno no sólamente para inflamar el ánimo del rey, mas aun para abrir el camino de fazer la guerra contra los romanos; y dizían que solo era uno el camino si passassen las armas en Ytalia y se asoldadassen guerreros del linaje ytálico, con los quales solos la provincia vencedora de otras gentes podría ser vençida. 5 Pidía  {\persName Hanníbal} al rey flota de çient naves y diez y seys mill peones y mill de cavallo, ofrecíase con estas compañas entrar en Ytalia y turbar a las gentes d’ella por tener sabido que por la reziente menoría de la guerra púnica que se les fizo, tenían muy aborrecido y temido el nombre de  {\persName Hanníbal}. 6 Otrosí tenía esperança de renovar la guerra púnica, si por el rey se le diesse licençia para embiar a  {\placeName Carthago} quien concitasse el vando Barchino y mover a sus amigos {\small\itshape [31amigos: traducción errónea del lat. \textit{animos}.]}  que no podían veer el imperio del pueblo romano. 7 El rey, consentiendo en todo aquesto, eligió a Aristón de Tyro, un ombre malicioso y cauteloso e ydóneo para tratar este negocio, y con grandes ofrescimientos y galardones fizo con él que llegasse a  {\placeName Carthago} y communicasse con los amigos y les diesse a entender en sus fablas algunas cosas que levava encomendadas.
\pend
\pstart
84.1 D’esta manera  {\persName Hanníbal}, desterrado de su patria, por todo el çircuyto de las tierras incitava guerra contra % image:../public/images/1491/179v.png
[179v,b] los romanos; y aquestos sus pensamientos ya parecían no ser vanos si Antíocho, segund avía començado, después le diera más fe que a sus \edtext{lisonjeros}{\Afootnote{; linsoieros ()}} y vestidos de púrpura. 2 Mas la invidia que las casas reales por la mayor parte crían, parió adversarios a  {\persName Hanníbal}, los quales, temiendo que tan cauteloso capitán, si aquistasse la gracia del rey con estos consejos, subiesse en muy alto grado de auctoridad y poderío, contendían de le fazer sospechoso. 3 Assí mesmo, en el mesmo tiempo acaesçió que Publio Villio, embaxador romano, vino a Épheso y ovo muchas fablas con  {\persName Hanníbal}, donde resultó ligera facultad a los adversarios maldizientes de le acriminar, y nasció en el ánimo de Anthíoco tanta sospecha de la fe de  {\persName Hanníbal}, que del todo fue excludido del consejo real.
\pend
\pstart
85.1 Hay quien diga que en mesmo tiempo que Publio Africano, uno de los embaxadores embiado a Anthíoco, ovo fabla con  {\persName Hanníbal} y que entre otras muchas cosas le demandó que, segund en su ánimo entendía, assí le dixesse pareçer quien judgava el que fasta entonces oviesse sido más principal capitán; 2 y respondidó  {\persName Hanníbal} que le parecía aver sido primero el rey de los macedones Alexandro, y el segundo Pyrrho, rey de los epirotas, y el mesmo  {\persName Hanníbal} se collocava por el terçero. 3 Y que Scipión Africano, sonriéndose mansamente a esto, le dixo: «¿Qué judgarías, o  {\persName Hanníbal}, si me vencieras?» Respondió: «Sin dubda me anteposiera a todos los otros capitanes». 4 Dizen que esta respuesta plugo a Scipión, que vio como el ingenio púnico ni le posponía, nin le aduzía en comparaçión, mas le dexava por manera de una occulta y conformada lisonja como a incomparable.
\pend
\pstart
86.1 Después d’esto,  {\persName Hanníbal} falló occasión de fablar con Anthíoco y, repetiendo desde su primera puericia todas las cosas passadas, se pudo desculpar de % image:../public/images/1491/180r.png
[180r,a] tal manera declarando su enemistad contra los romanos, que bolvió en la primera gracia y amistad del rey, que ya quasi tenía perdida. 2 Assí qu’el rey tenía assentado en su voluntad embiar en Ytalia a éste por capitán con la flota real, por experimentar el ánimo e ingenio del muy prestante varón y perpetuo enemigo del pueblo romano, segund lo tenía creýdo. 3 Mas uno llamado Thoas, príncipe de los Étholos, o por invidia o porque assí pensasse que se deviesse fazer, le fue contrario y fizo mudar la voluntad del rey, y del todo pudo destruyr aquesta deliberación que parecía induzir grande importancia para la guerra. 4 Ca exhortó a Anthíoco que passasse en Grecia y él mesmo por sí fiziesse sus negocios y no pareciesse que la gloria d’esta guerra se podía attribuyr a otri. El rey, dando fe a esto, dende a poco passó en Grecia para incitar la guerra contra los romanos.
\pend
\pstart
87.1 Y ende, después de passado no luengo espaçio de tiempo, tractando por común acuerdo de recebir en su estrecha amistad la gente de los thessalos, preguntado  {\persName Hanníbal} por su nombre que dixiesse en ello su parecer, de tal manera y con ánimo tan altivo, discutió no solamente del negocio de los thessalos, mas aun de la suma de las cosas, que conmovió la aprobatión y \edtext{consentimiento}{\Afootnote{; consentamiento ()}} de todos los que ende eran presentes. 2 Diziendo que no convenía mucho trabajar en lo de los thessalos, mas en todas manera devrían curar, segund su parecer, que Philippo, rey de Macedonia, veniesse a la compañía de la guerra o, no se llegando a alguna de las partes, estoviesse quedo y mirasse la contienda.
\pend
\pstart
88.1 Allende d’esto añadió el consejo de guerrear contra los romanos de çerca y ofreció y pronunció con muy extendidas palabras su obra para lo fazer. Fue oýdo % image:../public/images/1491/180r.png
[180r,b]  {\persName Hanníbal} con soberana attentión y su sentencia fue entonces allí más loada que approvada después por el effecto. 2 D’esta causa algunos y muchos se maravillan que aqueste \edtext{capitán}{\Afootnote{; copitan ()}} que tantos años contendiera con el pueblo romano quasi vencedor de todas las gentes, el rey lo pospusiesse en tiempo que mucho más devía requerir su obra y consejo. 3 ¿Quál capitán más cauto se podiera fallar en todo el çircuyto de las tierras? ¿Quién más accomodado y perteneciente para fazer guerra con los romanos? Con todo, el rey le pospuso en el comienço de las cosas que se avían de fazer y, no mucho después le puso en olvidança quando, escarneciendo de los consejos de todos los otros, aún confessava que uno solo,  {\persName Hanníbal}, avía antes visto las cosas que aprovechassen. 4 Ca la guerra fecha en Grecia, succediendo prósperamente a los romanos, partiose Anthíoco de Europa y acogiose a Épheso. Y allí, quito de cuydados, endresçava su ánimo a la paz, no creyendo que en Asia se deviessen temer las armas de los romanos. 5 Y no faltava lisonjeros que concertassen con él que tenía aquel tal deseo. Lo qual es mal perpetuo de los reyes que, oyendo de buena gana lo que quieren, no sienten que les dizen lisonjas y de grado padecen ser engañados. 6 Pero  {\persName Hanníbal}, que conoscía la potencia de los romanos y la cobdicia que tenían de señorear, amonestava al rey que en todo lo ál pensasse, salvo en la paz, y creyesse que nunca los romanos folgarían fasta que experimentassen si segund en  {\placeName África} y en Europa, assí mesmo en la otra tercia parte del circuyto de las tierras podiessen extender los fines del imperio.
\pend
\pstart
89.1 Movido por la auctoridad de  {\persName Hanníbal}, Antíocho luego embió a Polixenida, ombre sagaz y costumbrado a las % image:../public/images/1491/180v.png
[180v,a] peleas marítimas, y mandole salir al recuentro de la flota de los romanos que venía, y embió a  {\persName Hanníbal} que fuesse a Siria y aparejasse gran fuerça de navíos. 2 Y dio la capitanía d’esta a su flota al mesmo  {\persName Hanníbal} y a Apollonio, uno de sus purpurados, los quales capitanes, oyendo que Polixenida avía peleado con los romanos y no avía sido bien dichoso en ello, determinaron pelear con los rhodios, compañeros del pueblo romano. 3 En aquella contienda,  {\persName Hanníbal}, arremetiendo por la parte de la punta siniestra al capitán de los rhodios, Eudamo, avíale atajado su nave pretoria y sin dubda tenía gran mejoría; los enemigos que en la otra punta peleavan con Apollonio fiziéronle fuyr y volaron tan presto, que quitaron la victoria ya quasi çierta de las manos de  {\persName Hanníbal}.
\pend
\pstart
90.1 Después d’esta pelea naval poco dichosa, no sabemos que  {\persName Hanníbal} fiziesse cosa digna de memoria. Ca vençido Anthíoco, los romanos, allende de las otras condiciones que posieron al rey, demandavan que les entregasse a  {\persName Hanníbal}, perpetuo enemigo de su república. 2 Y  {\persName Hanníbal}, que mucho antes lo tenía considerado, luego después de aquella memorable batalla cometida çerca de Magnesia, donde descayeron los favores de Anthíoco, se partió d’él; 3 y después de luengos çircuytos y rodeos de caminos, al fin se recogió, fuyendo a Prusia, rey de Bithinia, non porque reposiesse mucha esperança en su amistad, mas porque señoreando los romanos extendidamente por tierra y por mar por el mundo, le parecía deverse acojer al logar más necessario que podiesse quando no le fallasse tan seguro como quesiesse.
\pend
\pstart
91.1 Hay quien diga que, vencido Anthíoco,  {\persName Hanníbal} se fue en Creta a los cortynos y que allí luego fue divulgado que traýa consigo grand suma de oro y de plata. Y temiendo que los cretenses por esto % image:../public/images/1491/180v.png
[180v,b] le echassen la mano, dizen que falló tal remedio para evitar el peligro: 2 Mandó poner en el templo de Diana cántaros llenos de plomo sobre dorado, fingiendo ser muy solícito y tener grand cuydado de la guarda de las vasijas {\small\itshape [32vasijas del tesoro: traducción por amplificación del lat. \textit{thesauri}.]}  del tesoro. Y de la otra parte, puso en casa estatuas de metal llenas de dinero. Y mientra que ellos guardavan el templo, que no sacassen dende cosa {\small\itshape [33cosa: traducción del lat. \textit{amphorae} por un término genérico.]}  sin su sabidoría d’ellos,  {\persName Hanníbal} fizo vela y fuese a Bithinia. 3 Hay en Bithinia un barrio o pequeño pueblo junto a la ribera del mar que los moradores de aquella comarca llaman Libyssa, del qual hay versos divulgados que suelen contar en diversas partes: «La tierra Libyssa dio sepulcro al cuerpo de  {\persName Hanníbal}». En aquel logar fazía morada  {\persName Hanníbal} sin passar la vida ociosa, mas ponía estudio en disponer los marineros y en exercitar su gente y tratar cavallos.
\pend
\pstart
92.1 Algunos scriptores dizen que en el mesmo tiempo Prusia tenía guerra con Eumene, rey de Pérgamo, que era compañero y amigo romano, y Prusia fizo capitán de su flota a  {\persName Hanníbal}, y que acometió a Eumene, con nuevo engaño \edtext{donde}{\Afootnote{; dondo ()}} pudo reportar victoria en la batalla naval. 2 Ca antes que començassen la pelea, dizen que  {\persName Hanníbal} tenía metida en vasijas de tierra grand quantidad de serpientes, y començada ya la batalla, quando todos tenían los ojos y los ánimos occupados en pelear, lançó las vasijas dentro de los navíos de los enemigos y con esto, rebueltos los enemigos y espantados a causa de la novedad, bolvieron fuyendo. 3 D’esta fazaña no fazen mentión los anales más antiguos, salvo Emilio y Trogo. La fe d’ello consiste çerca de los auctores. 4 Denunciadas las contiendas d’estos dos reyes en Roma, el senado embió por embaxador en Asia a Tito Flaminio que por sus nobles fazañas en Grecia pudo % image:../public/images/1491/181r.png
[181r,a] alcanzar muy honrado nombre, para que, segund puedo entender por conjectura, assentasse la paz entre los dos reyes.
\pend
\pstart
93.1 Aqueste Flaminio, venido al rey Prusia, ovo grande enojo que un ombre, tan enemigo del nombre romano, después de domadas y muertas tantas gentes y pueblos destruydos, aun remaneçiesse y con grand contienda ovo de impetrar del rey que le entregrasse a  {\persName Hanníbal}. 2 El qual desd’el comienço tenía sospecha de la liviandad de Prusia, y fiziera muchas minas en un su retraymiento de la morada, y tenía aparejadas siete salidas para poderse fuyr quan alguna súbita neçessidad le compeliesse. 3 Y la venida de Flaminio fizo creçer aquesta su no mediana sospecha, teniéndole por muy enemigo, no solamente por la común enemistad pública de los romanos contra él, mas aun a parte por la memoria de su padre Flaminio que fue muerto junto al lago de Transimeno, de manera que  {\persName Hanníbal} creýa serle este Flaminio más enemigo que otro alguno de todos los romanos.
\pend
\pstart
94.1 Assí que a cuydado con tan ansiosos cuydados, avía aparejado antes algunos remedios, segund diximos, para se fuyr, que le aprovechan a poco contra tan grandes contrastes y favores. Ca los guerreros de la guarda del rey fueron en mucha quantidad embiados a le prender y de tal manera çercaron la casa que, sentiendo  {\persName Hanníbal} la primera llegada d’ellos, tentó fuyr por la salida más occulta. 2 Pero quan conosçió que aquel logar estava occupado con guardas, ya desechada toda esperança de escaparse, determinó fuyr las manos de los romanos tan enemigables con muerte voluntaria. 3 Dizen algunos que por su mandado un siervo suyo le apretó la garganta y otros escriven que bevió sangre de toro, segund que Clitharco Ystratocle fingen de Themístocle y que con aquel bebraje cayó muerto. 4 Mas  {\persName Livio} , muy mucho rico auctor de historia, % image:../public/images/1491/181r.png
[181r,b] escrive que  {\persName Hanníbal} demandó una ponçoña que tenía él apparejada para semejantes casos y que, teniendo el vaso d’ella {\small\itshape [34vaso d’ella: en el sentido de ‘vaso de la ponzoña’.]}  mortal en la mano para la bever, dixo: «Quitemos tan grand cuydado a los romanos, pues que tanto deseo han tenido y al presente tienen de la muerte de un viejo ya consumido». 5 Los padres romanos amonestaron a Pyrrho, rey de los Epirotas, çercano a los muros de  {\placeName Roma} con señas enemigables que se guardasse de la ponçoña. Y éstos mesmos causaron que Prusia, huésped de  {\persName Hanníbal}, puesta en olvido la dignidad real y la diestra que avía juntado segurando a su hospedado  {\persName Hanníbal} {\small\itshape [35Prusia, huésped ... hospedado Hanníbal: traducción compleja que busca explicitar y diferenciar los términos latinos, por un lado, hospes, traducido como Prusia, huésped de Hanníbal y, por otro, hospitem traducido como su hospedado Hanníbal.]}  le entregasse malvadamente. 6 Assí que  {\persName Hanníbal}, dichas las palabras suso contenidas, con muchas maldiçiones contra Prusia, se mató con aquella ponçoña de edad de setenta años, segund fazen memoria algunos en sus escripturas. Y dizen qu’el cuerpo del defuncto fue colocado en Libyssa en un sepulcro de marmor, en que estava escripto: «Aquí está sepelido  {\persName Hanníbal}».
\pend
\pstart
95.1 Muchos {\small\itshape [36P. omite \textit{Sed eius mortis ... erat interpretabantur}.]}  reprehendían a Flaminio de crueldad, que assí como por gloria de una muy clara fazaña que oviesse conseguido, fuesse auctor de opprimir por tal manera a un varón tan anciano, de quien en lo de adelante no devía temer peligro la república romana, ya quasi vencedora de todas las gentes. 2 Otros algunos, excusando lo fecho, aun loavan a Flaminio en quitar de medio un tan perpetuo enemigo del pueblo romano, pues que, si le faltavan las fuerças del cuerpo en la edad robusta, no le falleçía ingenio y consejo y enseñança del exercicio militar, con los quales aparejos pudo incitar a Prusia a las armas y commover y perturbar a Asia con nuevas guerras. 3 Ca por el mesmo tiempo eran tan extendidas las riquezas del reyno de Bithynia, que pareçía no se dever assí ligeramente tener en poco, pues que en los tiempos sucedientes % image:../public/images/1491/181v.png
[181v,a] Mithridates, rey de Bithynia, luengamente con compañas navales y terrestres pudo fatigar al pueblo romano y ovo osadía de pelear en batalla contra Luçio Lúcullo y contra Gneo Pompeyo, tan prinçipales capitanes que tenían tan grandes exércitos. Aquello mesmo podieran temer del rey Prusia, mayormente capitaneando  {\persName Hanníbal}. 4 Assí que algunos piensan que la causa prinçipal porque Flaminio fue embiado embaxador al rey Prusia, fue por tractar la muerte de  {\persName Hanníbal} con secretos consejos. Con todo, se deve creer que Quinto Flaminio andovo buscando aquesto no tanto por quitar de medio por la presente muerte a  {\persName Hanníbal}, como a ombre que tan grandes daños y males avía fecho a su república, como por lo traer bivo a Roma, estimándose provechoso al pueblo romano y que él d’esto alcançaría grande honra.
\pend
\pstart
96.1 De tal linaje de muerte feneçió  {\persName Hanníbal}, varón carthaginés, sin dubda muy mucho excellente en los loores bélicos, por dexar las otras cosas, cuyo ánimo o ingenio o la singular disciplina en el exercicio militar se puede ligeramente entender de quánto momento fuessen en los negoçios, que los carthegineses, reçebida la guerra con tanta contención, 2 no confessaron primero ser vençidos, fasta que fue vençido  {\persName Hanníbal} en aquella gran batalla çerca de Zama, como si todas las fuerças de los que tenían en la guerra so capitanía de tal varón juntamente pareçiessen aver caýdo con su capitán. Feneçe la vida de  {\persName Hanníbal}
\pend

\endnumbering

\end{document}
